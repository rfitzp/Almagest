\chapter{Planetary Latitudes}
% !TEX root = ../Almagest.tex

\section{Introduction}
Up to now, we have neglected the fact that  the orbits of the
five visible planets about the Sun are all slightly inclined to the plane of the ecliptic. Of course, these inclinations cause
the ecliptic latitudes of the said planets to take small, but non-zero, values. 
In the following, we shall outline a model which is capable of predicting these values.

\section{Determination of Ecliptic Latitude of Superior Planet}
Figure~\ref{flong} shows the orbit of a superior planet. 
As we have already mentioned, the deferent and epicycle of such a planet have the same elements as 
the orbit of the planet in question  around the Sun, and the apparent orbit of the
Sun around the Earth, respectively. It follows that  the deferent and epicycle of a
superior planet are, respectively,  inclined  and parallel to the ecliptic plane.
 (Recall that the ecliptic plane corresponds to the
plane of the Sun's apparent orbit about the Earth.) Let the plane of the deferent cut the
ecliptic plane along the line $NGN'$. Here, $N$ is the point at which the
deferent passes through the  plane of the ecliptic from south to north, in the direction of the
mean planetary motion. This point is called the {\em ascending node}. 
Note that the line $NGN'$ must pass through point $G$, because the
Earth is common to the plane of the deferent and the ecliptic plane.
Now, it follows from simple geometry that the elevation of the guide-point $G'$ above  the
ecliptic plane satisfies $v= r\,\sin i\,\sin F$, where $r$ is the length $GG'$, $i$  the fixed inclination of the
planetary orbit (and, hence, of the deferent) to the ecliptic plane, and $F$  the angle $NGG'$. The
angle $F$ is termed the {\em argument of latitude}. We can write (see Chapter~\ref{csup})
\begin{equation}
F = \bar{F} + q,
\end{equation}
where $\bar{F}$ is the {\em mean argument of latitude}, and $q$ the equation of center of the deferent. Note that $\bar{F}$
increases uniformly in time; that is, 
\begin{equation}
\bar{F}  = \bar{F}_0 + \breve{n}\,(t-t_0).
\end{equation}
 Now, because the epicycle is parallel to the ecliptic
plane, the elevation of the planet above the said plane is
the same as that of the guide-point. Hence, from simple geometry, the ecliptic latitude of the planet satisfies
\begin{equation}
\beta = \frac{v}{r''},
\end{equation}
where $r''$ is the length $GP$, and
we have used the small angle approximation. However, it is apparent from Figure~\ref{vf5xx} that
\begin{equation}
r'' = (r^2 + 2\,r\,r'\,\cos\mu+ r'^{\,2})^{1/2},
\end{equation}
where $r'$ the length $G'P$, and $\mu$ the
equation of the epicycle. But, according to the analysis in Chapter~\ref{csup}, $r/r' = a\,z$, where $a$ is the planetary
major radius in units in which the major radius of the Sun's apparent orbit
about the Earth is unity, and $z$ is defined in Eq.~(\ref{ezdef}).
Thus, we obtain
\begin{equation}
\beta =h\,\beta_0,
\end{equation}
where
\begin{equation}
\beta_0(F) = \sin i\,\sin F
\end{equation}
is termed the {\em  deferential latitude},
and
\begin{equation}
h(\mu,z) = \left[1 + 2\,(a\,z)^{-1}\,\cos\mu+ (a\,z)^{-2}\right]^{-1/2}
\end{equation}
the {\em epicyclic latitude correction factor}.

\begin{figure}[h]
\centerline{\includegraphics[height=3in]{epsfiles/long.eps}}
\caption{Orbit of a superior planet. Here, $G$, $G'$, $P$, 
$N$, $N'$,  and $F$ represent the Earth,  the guide-point,
the planet, the ascending node, the descending node, and the argument of latitude, respectively. The view is from northern
ecliptic pole.}\label{flong}
\end{figure}

In the following, $a$, $e$, $n$, $\tilde{n}$, $\breve{n}$, $\bar{\lambda}_0$, $M_0$,  $\bar{F}_0$, and $i$ are  elements of the orbit of the planet in question
about the Sun, and $e_S$, $\zeta_S$, and $\lambda_S$  are elements of the Sun's apparent orbit
about the Sarth. 
The requisite elements for all of the superior planets at the J2000 epoch ($t_0=2\,451\,545.0$ JD)
are listed in Tables~\ref{lt4} and \ref{tadd}.
Employing a quadratic interpolation scheme to represent
$F(\mu,z)$ (see Chapter~\ref{csup}), our procedure for determining the ecliptic latitude of a
superior planet is summed up by the following formulae:
\begin{align}
\bar{\lambda}&=  \bar{\lambda}_0 + n\,(t-t_0) ,\\[0.5ex]
M &= M_0  +\tilde{n}\,(t-t_0),\\[0.5ex]
\bar{F} &= \bar{F}_0 + \breve{n}\,(t-t_0),\\[0.5ex]
q&= 2\,e\,\sin \,M + (5/4)\,e^2\,\sin\,2M,\\[0.5ex]
\zeta &= e\,\cos M - e^2\,\sin^2 M,\\[0.5ex]
F &=\bar{F} + q,\\[0.5ex]
\beta_0  &= \sin i\,\sin F,\\[0.5ex]
\mu&= \lambda_S - \bar{\lambda}-q,\\[0.5ex]
\bar{h} &= h(\mu,\bar{z})\equiv\left[1 + 2\,(a\,\bar{z})^{-1}\,\cos\mu+ (a\,\bar{z})^{-2}\right]^{-1/2},\\[0.5ex]
\delta h_- &= h(\mu,\bar{z}) - h(\mu,z_{\rm max}),\\[0.5ex]
\delta h_+&=h(\mu,z_{\rm min}) - h(\mu,\bar{z}),\\[0.5ex]
z &= \frac{1-\zeta}{1-\zeta_S},\\[0.5ex]
\xi &= \frac{\bar{z}-z}{\delta z},\\[0.5ex]
h  &=\Theta_-(\xi)\,\delta h_-+ \bar{h}
+ \Theta_+(\xi)\,\delta\,h_+,\\[0.5ex]
\beta &= h\,\beta_0.
\end{align}
Here, $\bar{z} = (1+e\,e_S)/(1-e_S^{\,2})$, $\delta z = (e+e_S)/(1-e_S^{\,2})$, $z_{\rm min} = \bar{z}-\delta z$,
and $z_{\rm max} = \bar{z}+\delta z$. The constants $\bar{z}$, $\delta z$, $z_{\rm min}$, and $z_{\rm max}$ 
for each of the superior planets are listed in Table~\ref{vtx}.  Finally, the functions $\Theta_\pm$ are  tabulated in Table~\ref{vty}.

For the case of Mars, the previous formulae are capable of matching NASA ephemeris data during the years 1995--2006 AD
with a mean error of $0.3'$ and a maximum error of $1.5'$. For the case of Jupiter, the mean error is
$0.2'$ and the maximum error $0.5'$. Finally, for the case of Saturn, the mean error is $0.05'$ and the
maximum error $0.08'$. 

\section{Mars}
The ecliptic latitude of Mars can be determined with the aid of Tables~\ref{vt8}, \ref{tlat1m}, and \ref{tlat2m}. Table~\ref{vt8} allows
the mean argument of latitude, $\bar{F}$, of Mars to be calculated as a function of
time. Next, Table~\ref{tlat1m} permits the deferential latitude, $\beta_0$, to
be determined as a function of the true argument of latitude, $F$. Finally, Table~\ref{tlat2m} allows the quantities
$\delta h_-$, $\bar{h}$, and $\delta h_+$ to be calculated as functions of the epicyclic
anomaly, $\mu$. 

The procedure for using the tables is as follows:
\begin{enumerate}

\item Determine the fractional Julian day number, $t$, corresponding to the date and time
at which the ecliptic latitude is to be calculated with the aid of Tables~\ref{kt1}--\ref{kt3}. Form $\Delta t = t-t_0$, where $t_0=2\,451\,545.0$ is the epoch. 
\
item Calculate the planetary equation
of center, $q$, ecliptic anomaly, $\mu$, and 
interpolation parameters $\Theta_+$ and $\Theta_-$ using the
procedure set out in Chapter~\ref{csup}.

\item Enter Table~\ref{vt8} with the digit for each power of 10
in ${\Delta} t$ and take out the corresponding values of $\Delta\bar{F}$. If $\Delta t$ is negative then the corresponding
values are also negative.
The value of the mean argument of latitude, $\bar{F}$, is the
sum of all the $\Delta\bar{F}$ values plus the value of $\bar{F}$ at the epoch.  

\item Form the true argument of latitude,
$F=\bar{F} + q$. Add as many multiples of $360^\circ$ to $F$ 
as is required to make it fall in the range $0^\circ$ to $360^\circ$.
Round $F$ to the nearest degree.

\item Enter Table~\ref{tlat1m} with the value of $F$ and take out the
corresponding value of the deferential latitude, $\beta_0$. It is necessary to interpolate if $F$ is odd.

\item Enter Table~\ref{tlat2m} with the value of $\mu$ and take
out the corresponding values of $\delta h_-$, $\bar{h}$, and
$\delta h_+$. If $\mu > 180^\circ$ then it is necessary to make use
of the identities $\delta h_\pm(360^\circ - \mu) = \delta h_\pm(\mu)$
and $\bar{h}(360^\circ - \mu) = \bar{h}(\mu)$.

\item Form the epicyclic latitude correction factor, $h = \Theta_-\,\delta h_-+ \bar{h}
+ \Theta_+\,\delta h_+$.

\item The ecliptic latitude, $\beta$, is the product of the deferential latitude, 
$\beta_0$,  and the epicyclic latitude correction factor, $h$.  The decimal fraction can
be converted into arc minutes
using Table~\ref{lt6a}. Round to the nearest arc minute. 
\end{enumerate}
One example of this procedure is given below.

~\\
\noindent {\em Example}: May 5, 2005 AD, 00:00 UT:\\
~\\
From Cha.~\ref{csup}, $t-t_0=1\,950.5$ JD, 
$q= -7.345^\circ$,  $\mu= 114.286^\circ$, $\Theta_-=0.101$, and
$\Theta_+ = 0.619$.
 Making use of
Table~\ref{vt8}, we find:\\
\begin{tabular}{rr}
&\\
$t$(JD) & $\bar{F}(^\circ)$\\[-2ex]
&\\
+1000 & $164.041$ \\
+900 & $111.637$\\
+50 & $26.202$ \\
+.5 & $0.262$\\
Epoch & $305.796$ \\\cline{2-2}
&$607.938$ \\\cline{2-2}
Modulus & $247.938$ \\ 
&\\
\end{tabular}\\
Thus,
$$
F = \bar{F} + q = 247.938-7.345 = 240.593\simeq 241^\circ.
$$
It follows from Table~\ref{tlat1m} that
$$
\beta_0(241^\circ) = -1.615^\circ.
$$
Since $\mu\simeq 114^\circ$, Table~\ref{tlat2m} yields
$$
\delta h_-(114^\circ) = -0.017,\mbox{\hspace{0.5cm}}\bar{h}(114^\circ)=1.056, \mbox{\hspace{0.5cm}}\delta h_+(114^\circ) = -0.027,
$$
so
$$
h = \Theta_-\,\delta h_- + \bar{h}+\Theta_+\,\delta h_+ = -0.101\times 0.017+1.056-0.619\times 0.027 = 1.038.
$$
Finally,
$$
\beta = h\,\beta_0 = -1.038\times 1.615 = -1.676 \simeq -1^\circ 41'.
$$
Thus,
the ecliptic latitude of Mars at 00:00 UT on May 5, 2005 AD was $-1^\circ 41'$.

\section{Jupiter}
The ecliptic latitude of Jupiter can be determined with the aid of Tables~\ref{vt11}, \ref{tlat1j}, and \ref{tlat2j}. Table~\ref{vt11} allows
the mean argument of latitude, $\bar{F}$, of Jupiter to be calculated as a function of
time. Next, Table~\ref{tlat1j} permits the deferential latitude, $\beta_0$, to
be determined as a function of the true argument of latitude, $F$. Finally, Table~\ref{tlat2j} allows the quantities
$\delta h_-$, $\bar{h}$, and $\delta h_+$ to be calculated as functions of the epicyclic
anomaly, $\mu$.  The procedure for using these tables is analogous to the previously described procedure for
using the Mars tables.
One example of this procedure is given below.

~\\
\noindent {\em Example}: May 5, 2005 AD, 00:00 UT:\\
~\\
From Cha.~\ref{csup}, $t-t_0=1\,950.5$ JD, 
$q= -0.091^\circ$,  $\mu= 208.192^\circ$, $\Theta_-=-0.469$, and
$\Theta_+ = -0.121$.
 Making use of
Table~\ref{vt11}, we find:\\
\begin{tabular}{rr}
&\\
$t$(JD) & $\bar{F}(^\circ)$\\[-2ex]
&\\
+1000 & $83.081$ \\
+900 & $74.773$\\
+50 & $4.154$ \\
+.5 & $0.042$\\
Epoch & $293.660$ \\\cline{2-2}
&$455.710$ \\\cline{2-2}
Modulus & $95.710$ \\ 
&\\
\end{tabular}\\
Thus,
$$
F = \bar{F} + q =95.710-0.091 = 95.619\simeq 96^\circ.
$$
It follows from Table~\ref{tlat1j} that
$$
\beta_0(96^\circ) = 1.297^\circ.
$$
Since $\mu\simeq 208^\circ$, Table~\ref{tlat2j} yields
$$
\delta h_-(208^\circ) = 0.014,\mbox{\hspace{0.5cm}}\bar{h}(208^\circ)=1.197, \mbox{\hspace{0.5cm}}\delta h_+(208^\circ) = 0.016,
$$
so
$$
h = \Theta_-\,\delta h_- + \bar{h}+\Theta_+\,\delta h_+ = -0.469\times 0.014 +1.197-0.121\times 0.016 = 1.188.
$$
Finally,
$$
\beta = h\,\beta_0 = 1.188\times 1.297 = 1.541 \simeq 1^\circ 32'.
$$
Thus,
the ecliptic latitude of Jupiter at 00:00 UT on May 5, 2005 CE was $1^\circ 32'$.

\section{Saturn}
The ecliptic latitude of Saturn can be determined with the aid of Tables~\ref{vt14}, \ref{tlat1s}, and \ref{tlat2s}. Table~\ref{vt14} allows
the mean argument of latitude, $\bar{F}$, of Saturn to be calculated as a function of
time. Next, Table~\ref{tlat1s} permits the deferential latitude, $\beta_0$, to
be determined as a function of the true argument of latitude, $F$. Finally, Table~\ref{tlat2s} allows the quantities
$\delta h_-$, $\bar{h}$, and $\delta h_+$ to be calculated as functions of the epicyclic
anomaly, $\mu$.  The procedure for using these tables is analogous to the previously described procedure for
using the Mars tables.
One example of this procedure is given below.

~\\
\noindent {\em Example}: May 5, 2005 AD, 00:00 UT:\\
~\\
From Cha.~\ref{csup}, $t-t_0=1\,950.5$ JD, 
$q= 2.561^\circ$,  $\mu= 286.625^\circ$, $\Theta_-=0.071$, and
$\Theta_+ = 0.759$.
 Making use of
Table~\ref{vt14}, we find:\\
\begin{tabular}{rr}
&\\
$t$(JD) & $\bar{F}(^\circ)$\\[-2ex]
&\\
+1000 & $33.478$ \\
+900 & $30.130$\\
+50 & $1.674$ \\
+.5 & $0.017$\\
Epoch & $296.482$ \\\cline{2-2}
&$361.781$ \\\cline{2-2}
Modulus & $1.781$ \\ 
&\\
\end{tabular}\\
Thus,
$$
F = \bar{F} + q =1.781+2.561 = 4.342\simeq 4^\circ.
$$
It follows from Table~\ref{tlat1s} that
$$
\beta_0(4^\circ) = 0.173^\circ.
$$
Since $\mu\simeq 287^\circ$, Table~\ref{tlat2s} yields
$$
\delta h_-(287^\circ) = -0.002,\mbox{\hspace{0.5cm}}\bar{h}(287^\circ)=0.966, \mbox{\hspace{0.5cm}}\delta h_+(287^\circ) = -0.003,
$$
so
$$
h = \Theta_-\,\delta h_- + \bar{h}+\Theta_+\,\delta h_+ = -0.071\times 0.002 +0.966-0.759\times 0.003 = 0.964.
$$
Finally,
$$
\beta = h\,\beta_0 = 0.964\times 0.173 = 0.167 \simeq 0^\circ 10'.
$$
Thus,
the ecliptic latitude of Saturn at 00:00 UT on May 5, 2005 CE was $0^\circ 10'$.

\section{Determination of Ecliptic Latitude of Inferior Planet}
Figure~\ref{flong1} shows the orbit of an inferior planet. 
As we have already mentioned, the epicycle and deferent of such a planet 
have the same elements as
the orbit of the planet in question  around the Sun, and the apparent orbit of the
Sun around the Earth, respectively. It follows that  the epicycle and deferent of an inferior planet are, respectively,  inclined  and parallel to the ecliptic plane.
 Let the plane of the epicycle cut the
ecliptic plane along the line $NG'N'$. Here, $N$ is the point at which the
epicycle passes through the  plane of the ecliptic from south to north, in the direction of the
mean planetary motion. This point is called the {\em ascending node}. 
Note that the line $NG'N'$ must pass through the guide-point, $G'$, since the
Sun (which is coincident with the guide-point)  is common to the plane of the planetary orbit and the ecliptic plane.
Now, it follows from simple geometry that the elevation of the planet $P$ above  the
guide-point, $G'$, satisfies $v= r'\,\sin i\,\sin F$, where $r'$ is the length $G'P$, $i$  the fixed inclination of the
planetary orbit (and, hence, of the epicycle) to the ecliptic plane, and $F$  the angle $NG'P$. The
angle $F$ is termed the argument of latitude. We can write (see Chapter~\ref{cinf})
\begin{equation}
F = \bar{F} + q,
\end{equation}
where $\bar{F}$ is the mean argument of latitude, and $q$ the equation of center of the epicycle. Note that $\bar{F}$
increases uniformly  in time; that is, 
\begin{equation}
\bar{F}  = \bar{F}_0 + \breve{n}\,(t-t_0).
\end{equation}
 Now, because the deferent is parallel to the ecliptic
plane, the elevation of the planet above the said plane is
the same as that of the planet above the guide-point. Hence, from simple geometry, the ecliptic latitude of the planet satisfies
\begin{equation}
\beta = \frac{v}{r''},
\end{equation}
where $r''$ is the length $GP$, and
we have used the small angle approximation. However, it is apparent from Fig.~\ref{vf5xx} that
\begin{equation}
r'' = (r^2 + 2\,r\,r'\,\cos\mu+ r'^{\,2})^{1/2},
\end{equation}
where $r$ the length $GG'$, and $\mu$ the
equation of the epicycle. But, according to the analysis in Chapter~\ref{cinf}, $r'/r = a/z$, where $a$ is the planetary
major radius in units in which the major radius of the Sun's apparent orbit
about the Earth is unity, and $z$ is defined in Eq.~(\ref{e182x}).
Thus, we obtain
\begin{equation}
\beta =h\,\beta_0,
\end{equation}
where
\begin{equation}
\beta_0(F) = a\,\sin i\,\sin F
\end{equation}
is termed the {\em epicyclic latitude},
and
\begin{equation}
h(\mu,z) = \left[z^2 + 2\,a\,z\,\cos\mu+ a^2\right]^{-1/2}
\end{equation}
the {\em deferential latitude correction factor}.

\begin{figure}[h]
\centerline{\includegraphics[height=3in]{epsfiles/long1.eps}}
\caption{Orbit of an inferior planet. Here, $G$, $G'$, $P$, 
$N$, $N'$,  and $F$ represent the earth, the guide-point,
the planet, the ascending node, the descending node, and the argument of latitude, respectively. The view is from northern
ecliptic pole.}\label{flong1}
\end{figure}

In the following, $a$, $e$, $n$, $\tilde{n}$, $\breve{n}$, $\bar{\lambda}_0$, $M_0$,  $\bar{F}_0$, and $i$ are  elements of the orbit of the planet in question
about the Sun, and $e_S$, $\zeta_S$, and $\lambda_S$  are elements of the Sun's apparent orbit
about the Sarth. 
The requisite elements for all of the superior planets at the J2000 epoch ($t_0=2\,451\,545.0$ JD)
are listed in Tables~\ref{lt4} and \ref{tadd}.
Employing a quadratic interpolation scheme to represent
$F(\mu,z)$ (see Cha.~\ref{csup}), our procedure for determining the ecliptic latitude of a
superior planet is summed up by the following formulae:
\begin{align}
\bar{\lambda}&=  \bar{\lambda}_0 + n\,(t-t_0) ,\\[0.5ex]
M &= M_0  +\tilde{n}\,(t-t_0),\\[0.5ex]
\bar{F} &= \bar{F}_0 + \breve{n}\,(t-t_0),\\[0.5ex]
q&= 2\,e\,\sin \,M + (5/4)\,e^2\,\sin\,2M,\\[0.5ex]
\zeta &= e\,\cos M - e^2\,\sin^2 M,\\[0.5ex]
F &=&\bar{F} + q,\\[0.5ex]
\beta_0  &= a\,\sin i\,\sin F,\\[0.5ex]
\mu&= \bar{\lambda}+q-\bar{\lambda}_S,\\[0.5ex]
\bar{h} &= h(\mu,\bar{z})\equiv\left[\bar{z}^2 + 2\,a\,\bar{z}\,\cos\mu+ a^2\right]^{-1/2},\\[0.5ex]
\delta h_- &=h(\mu,\bar{z}) - h(\mu,z_{\rm max}),\\[0.5ex]
\delta h_+&=h(\mu,z_{\rm min}) - h(\mu,\bar{z}),\\[0.5ex]
z &= \frac{1-\zeta_S}{1-\zeta},\\[0.5ex]
\xi &= \frac{\bar{z}-z}{\delta z},\\[0.5ex]
h  &=\Theta_-(\xi)\,\delta h_-+ \bar{h}
+ \Theta_+(\xi)\,\delta\,h_+,\\[0.5ex]
\beta &=h\,\beta_0.
\end{align}
Here, $\bar{z} = (1+e\,e_S)/(1-e^{2})$, $\delta z = (e+e_S)/(1-e^{2})$, $z_{\rm min} = \bar{z}-\delta z$,
and $z_{\rm max} = \bar{z}+\delta z$. The constants $\bar{z}$, $\delta z$, $z_{\rm min}$, and $z_{\rm max}$ 
for each of the inferior planets are listed in Table~\ref{vtx}.  Finally, the functions $\Theta_\pm$ are  tabulated in Table~\ref{vty}.

For the case of Venus, the previous formulae are capable of matching NASA ephemeris data during the years 1995--2006 AD
with a mean error of $0.7'$ and a maximum error of $1.8'$. For the case of Mercury, with the augmentations to the theory described in Chapter~\ref{cinf}, the mean error is
$1.6'$ and the maximum error $5'$. 

\section{Venus}
The ecliptic latitude of Venus can be determined with the aid of Tables~\ref{vt17}, \ref{tlat1v}, and \ref{tlat2v}. Table~\ref{vt17} allows
the mean argument of latitude, $\bar{F}$, of Venus to be calculated as a function of
time. Next, Table~\ref{tlat1v} permits the epicyclic  latitude, $\beta_0$, to
be determined as a function of the true argument of latitude, $F$. Finally, Table~\ref{tlat2v} allows the quantities
$\delta h_-$, $\bar{h}$, and $\delta h_+$ to be calculated as functions of the epicyclic
anomaly, $\mu$. 

The procedure for using the tables is as follows:
\begin{enumerate}

\item Determine the fractional Julian day number, $t$, corresponding to the date and time
at which the ecliptic latitude is to be calculated with the aid of Tables~\ref{kt1}--\ref{kt3}. Form $\Delta t = t-t_0$, where $t_0=2\,451\,545.0$ is the epoch. 

\item Calculate the planetary equation
of center, $q$,  ecliptic anomaly, $\mu$, and 
interpolation parameters $\Theta_+$ and $\Theta_-$ using the
procedure set out in Chapter~\ref{cinf}.

\item Enter Table~\ref{vt17} with the digit for each power of 10
in ${\Delta} t$ and take out the corresponding values of $\Delta\bar{F}$. If $\Delta t$ is negative then the corresponding
values are also negative.
The value of the mean argument of latitude, $\bar{F}$, is the
sum of all the $\Delta\bar{F}$ values plus the value of $\bar{F}$ at the epoch.  

\item Form the true argument of latitude,
$F=\bar{F} + q$. Add as many multiples of $360^\circ$ to $F$ 
as is required to make it fall in the range $0^\circ$ to $360^\circ$.
Round $F$ to the nearest degree.

\item Enter Table~\ref{tlat1v} with the value of $F$ and take out the
corresponding value of the epicyclic latitude, $\beta_0$. It is necessary to interpolate if $F$ is odd.

\item Enter Table~\ref{tlat2v} with the value of $\mu$ and take
out the corresponding values of $\delta h_-$, $\bar{h}$, and
$\delta h_+$. If $\mu > 180^\circ$ then it is necessary to make use
of the identities $\delta h_\pm(360^\circ - \mu) = \delta h_\pm(\mu)$
and $\bar{h}(360^\circ - \mu) = \bar{h}(\mu)$.

\item Form the deferential latitude correction factor, $h = \Theta_-\,\delta h_-+ \bar{h}
+ \Theta_+\,\delta h_+$.

\item The ecliptic latitude, $\beta$, is the product of the epicyclic latitude, 
$\beta_0$,  and the deferential latitude correction factor, $h$.  The decimal fraction can
be converted into arc minutes
using Table~\ref{lt6a}. Round to the nearest arc minute. 
\end{enumerate}
One example of this procedure is given below.

~\\
\noindent {\em Example}: May 5, 2005 AD, 00:00 UT:\\
~\\
From Cha.~\ref{cinf}, $t-t_0=1\,950.5$ JD, 
$q= -0.712^\circ$,  $\mu= 21.689^\circ$, $\Theta_-=-0.355$, and
$\Theta_+ = -0.125$.
 Making use of
Table~\ref{vt17}, we find:\\
\begin{tabular}{rr}
&\\
$t$(JD) & $\bar{F}(^\circ)$\\[-2ex]
&\\
+1000 & $162.138$ \\
+900 & $1.924$\\
+50 & $80.107$ \\
+.5 & $0.801$\\
Epoch & $105.253$ \\\cline{2-2}
&$350.223$ \\\cline{2-2}
Modulus & $350.223$ \\ 
&\\
\end{tabular}\\
Thus,
$$
F = \bar{F} + q = 350.223-0.712 = 349.511\simeq 350^\circ.
$$
It follows from Table~\ref{tlat1v} that
$$
\beta_0(350^\circ) = -0.423^\circ.
$$
Since $\mu\simeq 22^\circ$, Table~\ref{tlat2v} yields
$$
\delta h_-(22^\circ) = 0.008,\mbox{\hspace{0.5cm}}\bar{h}(22^\circ)=0.591, \mbox{\hspace{0.5cm}}\delta h_+(22^\circ) = 0.008,
$$
so
$$
h = \Theta_-\,\delta h_- + \bar{h}+\Theta_+\,\delta h_+ = -0.355\times 0.008+0.591-0.125\times 0.008 = 0.587.
$$
Finally,
$$
\beta = h\,\beta_0 =- 0.587\times 0.423 = -0.248\simeq -0^\circ 15'.
$$
Thus,
the ecliptic latitude of Venus at 00:00 UT on May 5, 2005 AD was $-0^\circ 15'$.

\section{Mercury}
The ecliptic latitude of Mercury can be determined with the aid of Tables~\ref{vt20}, \ref{tlat1mc}, and \ref{tlat2mc}. Table~\ref{vt20} allows
the mean argument of latitude, $\bar{F}$, of Mercury to be calculated as a function of
time. Next, Table~\ref{tlat1mc} permits the epicyclic  latitude, $\beta_0$, to
be determined as a function of the true argument of latitude, $F$. Finally, Table~\ref{tlat2mc} allows the quantities
$\delta h_-$, $\bar{h}$, and $\delta h_+$ to be calculated as functions of the epicyclic
anomaly, $\mu$. 
The procedure for using the tables is analogous to the previously
described procedure for using the Venus tables.
One example of this procedure is given below.

~\\
\noindent {\em Example}: May 5, 2005 AD, 00:00 UT:\\
~\\
From Cha.~\ref{cinf}, $t-t_0=1\,950.5$ JD, 
$q= -16.974^\circ$,  $\mu= 252.692^\circ$, $\Theta_-=0.107$, and
$\Theta_+ = 0.583$.
 Making use of
Table~\ref{vt20}, we find:\\
\begin{tabular}{rr}
&\\
$t$(JD) & $\bar{F}(^\circ)$\\[-2ex]
&\\
+1000 & $132.342$ \\
+900 & $83.108$\\
+50 & $204.617$ \\
+.5 & $2.046$\\
Epoch & $204.436$ \\\cline{2-2}
&$626.549$ \\\cline{2-2}
Modulus & $266.549$ \\ 
&\\
\end{tabular}\\
Thus,
$$
F = \bar{F} + q = 266.549-16.974= 249.575\simeq 250^\circ.
$$
It follows from Table~\ref{tlat1mc} that
$$
\beta_0(250^\circ) = -2.511^\circ.
$$
Since $\mu\simeq 253^\circ$, Table~\ref{tlat2mc} yields
$$
\delta h_-(253^\circ) = 0.184,\mbox{\hspace{0.5cm}}\bar{h}(253^\circ)=1.037, \mbox{\hspace{0.5cm}}\delta h_+(253^\circ) = 0.272,
$$
so
$$
h = \Theta_-\,\delta h_- + \bar{h}+\Theta_+\,\delta h_+ = 0.107\times 0.184+1.037+0.583\times 0.272= 1.215.
$$
Finally,
$$
\beta = h\,\beta_0 =- 1.215\times 2.511 = -3.051\simeq -3^\circ 03'.
$$
Thus,
the ecliptic latitude of Mercury at 00:00 UT on May 5, 2005 CE was $-3^\circ 03'$.

\clearpage
\begin{table}\centering
\begin{tabular}{l|ccc}
Object &  $i(^\circ)$& $\breve{n}\,(^\circ/{\rm day})$ & $\bar{F}_0\,(^\circ)$\\\hline
&&&\\[-2.2ex]
Mercury &  $6.9190$ & $4.09234221$ & $204.436$\\
Venus     &  $3.3692$ & $1.60213807$ & $105.253$ \\
Mars      &  $1.8467$  & $0.52404094$ & $305.796$ \\
Jupiter   &  $1.3044$  & $0.08308122$ & $293.660$ \\
Saturn   &  $2.4860$  & $0.03347795$ & $296.482$ \\
\end{tabular}
\caption{Additional Keplerian orbital elements for  the five visible planets at the J2000 epoch (that is, 12:00 UT, January 1, 2000 CE,
which corresponds to $t_0 = 2\,451\,545.0$ JD). The elements are optimized for use in the
time period 1800 CE to 2050 AD. Source: Jet Propulsion Laboratory (NASA), {\tt http://ssd.jpl.nasa.gov/}. }\label{tadd}
\end{table}

\newpage
\begin{table}\centering
\small{ \begin{tabular}{crc}
$F (^\circ)$ & $\beta_0(^\circ)$ &
$F (^\circ)$ \\\hline
&&\\[-1.75ex]
000/180 &  0.000 & (180)/(360)\\
002/178 &  0.064 & (182)/(358)\\
004/176 &  0.129 & (184)/(356)\\
006/174 &  0.193 & (186)/(354)\\
008/172 &  0.257 & (188)/(352)\\
010/170 &  0.321 & (190)/(350)\\
012/168 &  0.384 & (192)/(348)\\
014/166 &  0.447 & (194)/(346)\\
016/164 &  0.509 & (196)/(344)\\
018/162 &  0.571 & (198)/(342)\\
020/160 &  0.631 & (200)/(340)\\
022/158 &  0.692 & (202)/(338)\\
024/156 &  0.751 & (204)/(336)\\
026/154 &  0.809 & (206)/(334)\\
028/152 &  0.867 & (208)/(332)\\
030/150 &  0.923 & (210)/(330)\\
032/148 &  0.978 & (212)/(328)\\
034/146 &  1.032 & (214)/(326)\\
036/144 &  1.085 & (216)/(324)\\
038/142 &  1.137 & (218)/(322)\\
040/140 &  1.187 & (220)/(320)\\
042/138 &  1.235 & (222)/(318)\\
044/136 &  1.283 & (224)/(316)\\
046/134 &  1.328 & (226)/(314)\\
048/132 &  1.372 & (228)/(312)\\
050/130 &  1.414 & (230)/(310)\\
052/128 &  1.455 & (232)/(308)\\
054/126 &  1.494 & (234)/(306)\\
056/124 &  1.531 & (236)/(304)\\
058/122 &  1.566 & (238)/(302)\\
060/120 &  1.599 & (240)/(300)\\
062/118 &  1.630 & (242)/(298)\\
064/116 &  1.660 & (244)/(296)\\
066/114 &  1.687 & (246)/(294)\\
068/112 &  1.712 & (248)/(292)\\
070/110 &  1.735 & (250)/(290)\\
072/108 &  1.756 & (252)/(288)\\
074/106 &  1.775 & (254)/(286)\\
076/104 &  1.792 & (256)/(284)\\
078/102 &  1.806 & (258)/(282)\\
080/100 &  1.818 & (260)/(280)\\
082/098 &  1.828 & (262)/(278)\\
084/096 &  1.836 & (264)/(276)\\
086/094 &  1.842 & (266)/(274)\\
088/092 &  1.845 & (268)/(272)\\
090/090 &  1.846 & (270)/(270)\\
\end{tabular}}
\caption{Deferential ecliptic latitude of Mars.  The latitude is minus the value shown
in the table if the argument is
in parenthesies. }\label{tlat1m}
\end{table}

\newpage
\begin{table}\centering
\small{ \begin{tabular}{rrrr|rrrr|rrrr|crrr}
$\mu$ & $\delta h_-$  & $\bar{h}~~~~$ & $\delta h_+$ &
$\mu$ & $\delta h_-$  & $\bar{h}~~~~$ & $\delta h_+$ &
$\mu$ & $\delta h_-$  & $\bar{h}~~~~$ & $\delta h_+$ &
$\mu$ & $\delta h_-$  & $\bar{h}~~~~$ & $\delta h_+$ \\\hline
&&&&&&&&&&&&&&&\\[-1.75ex]
  0 & \tiny{ $-$0.025} &   0.604 & \tiny{ $-$0.028} &  45 & \tiny{ $-$0.025} &   0.652 & \tiny{ $-$0.029} &  90 & \tiny{ $-$0.025} &   0.836 & \tiny{ $-$0.031} & 135 & \tiny{  0.015} &   1.410 & \tiny{  0.003}\\
  1 & \tiny{ $-$0.025} &   0.604 & \tiny{ $-$0.028} &  46 & \tiny{ $-$0.025} &   0.654 & \tiny{ $-$0.029} &  91 & \tiny{ $-$0.025} &   0.843 & \tiny{ $-$0.031} & 136 & \tiny{  0.018} &   1.433 & \tiny{  0.006}\\
  2 & \tiny{ $-$0.025} &   0.604 & \tiny{ $-$0.028} &  47 & \tiny{ $-$0.025} &   0.656 & \tiny{ $-$0.029} &  92 & \tiny{ $-$0.024} &   0.850 & \tiny{ $-$0.031} & 137 & \tiny{  0.021} &   1.457 & \tiny{  0.009}\\
  3 & \tiny{ $-$0.025} &   0.604 & \tiny{ $-$0.028} &  48 & \tiny{ $-$0.025} &   0.659 & \tiny{ $-$0.029} &  93 & \tiny{ $-$0.024} &   0.857 & \tiny{ $-$0.031} & 138 & \tiny{  0.025} &   1.482 & \tiny{  0.013}\\
  4 & \tiny{ $-$0.025} &   0.605 & \tiny{ $-$0.028} &  49 & \tiny{ $-$0.025} &   0.661 & \tiny{ $-$0.029} &  94 & \tiny{ $-$0.024} &   0.865 & \tiny{ $-$0.031} & 139 & \tiny{  0.028} &   1.507 & \tiny{  0.017}\\
  5 & \tiny{ $-$0.025} &   0.605 & \tiny{ $-$0.028} &  50 & \tiny{ $-$0.025} &   0.664 & \tiny{ $-$0.030} &  95 & \tiny{ $-$0.024} &   0.872 & \tiny{ $-$0.031} & 140 & \tiny{  0.032} &   1.533 & \tiny{  0.021}\\
  6 & \tiny{ $-$0.025} &   0.605 & \tiny{ $-$0.028} &  51 & \tiny{ $-$0.025} &   0.666 & \tiny{ $-$0.030} &  96 & \tiny{ $-$0.024} &   0.880 & \tiny{ $$-$$0.031} & 141 & \tiny{  0.037} &   1.560 & \tiny{  0.026}\\
  7 & \tiny{ $-$0.025} &   0.605 & \tiny{ $-$0.028} &  52 & \tiny{ $-$0.025} &   0.669 & \tiny{ $-$0.030} &  97 & \tiny{ $-$0.024} &   0.888 & \tiny{ $-$0.031} & 142 & \tiny{  0.041} &   1.588 & \tiny{  0.031}\\
  8 & \tiny{ $-$0.025} &   0.606 & \tiny{ $-$0.028} &  53 & \tiny{ $-$0.025} &   0.672 & \tiny{ $-$0.030} &  98 & \tiny{ $-$0.023} &   0.896 & \tiny{ $-$0.031} & 143 & \tiny{  0.046} &   1.616 & \tiny{  0.036}\\
  9 & \tiny{ $-$0.025} &   0.606 & \tiny{ $-$0.028} &  54 & \tiny{ $-$0.025} &   0.674 & \tiny{ $-$0.030} &  99 & \tiny{ $-$0.023} &   0.904 & \tiny{ $-$0.031} & 144 & \tiny{  0.051} &   1.646 & \tiny{  0.043}\\
 10 & \tiny{ $-$0.025} &   0.606 & \tiny{ $-$0.028} &  55 & \tiny{ $-$0.025} &   0.677 & \tiny{ $-$0.030} & 100 & \tiny{ $-$0.023} &   0.912 & \tiny{ $-$0.030} & 145 & \tiny{  0.057} &   1.676 & \tiny{  0.049}\\
 11 & \tiny{ $-$0.025} &   0.607 & \tiny{ $-$0.028} &  56 & \tiny{ $-$0.025} &   0.680 & \tiny{ $-$0.030} & 101 & \tiny{ $-$0.023} &   0.921 & \tiny{ $-$0.030} & 146 & \tiny{  0.063} &   1.708 & \tiny{  0.056}\\
 12 & \tiny{ $-$0.025} &   0.607 & \tiny{ $-$0.028} &  57 & \tiny{ $-$0.026} &   0.683 & \tiny{ $-$0.030} & 102 & \tiny{ $-$0.022} &   0.930 & \tiny{ $-$0.030} & 147 & \tiny{  0.069} &   1.740 & \tiny{  0.064}\\
 13 & \tiny{ $-$0.025} &   0.608 & \tiny{ $-$0.028} &  58 & \tiny{ $-$0.026} &   0.686 & \tiny{ $-$0.030} & 103 & \tiny{ $-$0.022} &   0.939 & \tiny{ $-$0.030} & 148 & \tiny{  0.076} &   1.773 & \tiny{  0.073}\\
 14 & \tiny{ $-$0.025} &   0.609 & \tiny{ $-$0.028} &  59 & \tiny{ $-$0.026} &   0.689 & \tiny{ $-$0.030} & 104 & \tiny{ $-$0.022} &   0.948 & \tiny{ $-$0.030} & 149 & \tiny{  0.084} &   1.807 & \tiny{  0.083}\\
 15 & \tiny{ $-$0.025} &   0.609 & \tiny{ $-$0.028} &  60 & \tiny{ $-$0.026} &   0.693 & \tiny{ $-$0.030} & 105 & \tiny{ $-$0.021} &   0.958 & \tiny{ $-$0.030} & 150 & \tiny{  0.092} &   1.843 & \tiny{  0.093}\\
 16 & \tiny{ $-$0.025} &   0.610 & \tiny{ $-$0.028} &  61 & \tiny{ $-$0.026} &   0.696 & \tiny{ $-$0.030} & 106 & \tiny{ $-$0.021} &   0.968 & \tiny{ $-$0.029} & 151 & \tiny{  0.100} &   1.879 & \tiny{  0.104}\\
 17 & \tiny{ $-$0.025} &   0.611 & \tiny{ $-$0.028} &  62 & \tiny{ $-$0.026} &   0.699 & \tiny{ $-$0.030} & 107 & \tiny{ $-$0.021} &   0.978 & \tiny{ $-$0.029} & 152 & \tiny{  0.109} &   1.916 & \tiny{  0.117}\\
 18 & \tiny{ $-$0.025} &   0.611 & \tiny{ $-$0.028} &  63 & \tiny{ $-$0.026} &   0.703 & \tiny{ $-$0.030} & 108 & \tiny{ $-$0.020} &   0.988 & \tiny{ $-$0.029} & 153 & \tiny{  0.118} &   1.955 & \tiny{  0.130}\\
 19 & \tiny{ $-$0.025} &   0.612 & \tiny{ $-$0.028} &  64 & \tiny{ $-$0.026} &   0.706 & \tiny{ $-$0.030} & 109 & \tiny{ $-$0.020} &   0.999 & \tiny{ $-$0.029} & 154 & \tiny{  0.128} &   1.994 & \tiny{  0.145}\\
 20 & \tiny{ $-$0.025} &   0.613 & \tiny{ $-$0.028} &  65 & \tiny{ $-$0.026} &   0.710 & \tiny{ $-$0.030} & 110 & \tiny{ $-$0.019} &   1.010 & \tiny{ $-$0.028} & 155 & \tiny{  0.139} &   2.034 & \tiny{  0.161}\\
 21 & \tiny{ $-$0.025} &   0.614 & \tiny{ $-$0.028} &  66 & \tiny{ $-$0.026} &   0.714 & \tiny{ $-$0.030} & 111 & \tiny{ $-$0.019} &   1.021 & \tiny{ $-$0.028} & 156 & \tiny{  0.151} &   2.075 & \tiny{  0.178}\\
 22 & \tiny{ $-$0.025} &   0.615 & \tiny{ $-$0.028} &  67 & \tiny{ $-$0.026} &   0.718 & \tiny{ $-$0.030} & 112 & \tiny{ $-$0.018} &   1.032 & \tiny{ $-$0.027} & 157 & \tiny{  0.163} &   2.117 & \tiny{  0.197}\\
 23 & \tiny{ $-$0.025} &   0.616 & \tiny{ $-$0.028} &  68 & \tiny{ $-$0.026} &   0.722 & \tiny{ $-$0.030} & 113 & \tiny{ $-$0.018} &   1.044 & \tiny{ $-$0.027} & 158 & \tiny{  0.175} &   2.160 & \tiny{  0.218}\\
 24 & \tiny{ $-$0.025} &   0.617 & \tiny{ $-$0.028} &  69 & \tiny{ $-$0.026} &   0.726 & \tiny{ $-$0.031} & 114 & \tiny{ $-$0.017} &   1.056 & \tiny{ $-$0.027} & 159 & \tiny{  0.189} &   2.203 & \tiny{  0.240}\\
 25 & \tiny{ $-$0.025} &   0.618 & \tiny{ $-$0.029} &  70 & \tiny{ $-$0.026} &   0.730 & \tiny{ $-$0.031} & 115 & \tiny{ $-$0.016} &   1.069 & \tiny{ $-$0.026} & 160 & \tiny{  0.202} &   2.247 & \tiny{  0.265}\\
 26 & \tiny{ $-$0.025} &   0.619 & \tiny{ $-$0.029} &  71 & \tiny{ $-$0.026} &   0.734 & \tiny{ $-$0.031} & 116 & \tiny{ $-$0.015} &   1.082 & \tiny{ $-$0.025} & 161 & \tiny{  0.217} &   2.292 & \tiny{  0.291}\\
 27 & \tiny{ $-$0.025} &   0.621 & \tiny{ $-$0.029} &  72 & \tiny{ $-$0.026} &   0.738 & \tiny{ $-$0.031} & 117 & \tiny{ $-$0.015} &   1.095 & \tiny{ $-$0.025} & 162 & \tiny{  0.232} &   2.337 & \tiny{  0.319}\\
 28 & \tiny{ $-$0.025} &   0.622 & \tiny{ $-$0.029} &  73 & \tiny{ $-$0.026} &   0.743 & \tiny{ $-$0.031} & 118 & \tiny{ $-$0.014} &   1.108 & \tiny{ $-$0.024} & 163 & \tiny{  0.248} &   2.382 & \tiny{  0.349}\\
 29 & \tiny{ $-$0.025} &   0.623 & \tiny{ $-$0.029} &  74 & \tiny{ $-$0.026} &   0.747 & \tiny{ $-$0.031} & 119 & \tiny{ $-$0.013} &   1.122 & \tiny{ $-$0.023} & 164 & \tiny{  0.264} &   2.427 & \tiny{  0.382}\\
 30 & \tiny{ $-$0.025} &   0.625 & \tiny{ $-$0.029} &  75 & \tiny{ $-$0.026} &   0.752 & \tiny{ $-$0.031} & 120 & \tiny{ $-$0.012} &   1.137 & \tiny{ $-$0.023} & 165 & \tiny{  0.280} &   2.472 & \tiny{  0.416}\\
 31 & \tiny{ $-$0.025} &   0.626 & \tiny{ $-$0.029} &  76 & \tiny{ $-$0.026} &   0.757 & \tiny{ $-$0.031} & 121 & \tiny{ $-$0.011} &   1.151 & \tiny{ $-$0.022} & 166 & \tiny{  0.297} &   2.517 & \tiny{  0.453}\\
 32 & \tiny{ $-$0.025} &   0.627 & \tiny{ $-$0.029} &  77 & \tiny{ $-$0.026} &   0.762 & \tiny{ $-$0.031} & 122 & \tiny{ $-$0.010} &   1.167 & \tiny{ $-$0.021} & 167 & \tiny{  0.314} &   2.560 & \tiny{  0.491}\\
 33 & \tiny{ $-$0.025} &   0.629 & \tiny{ $-$0.029} &  78 & \tiny{ $-$0.025} &   0.767 & \tiny{ $-$0.031} & 123 & \tiny{ $-$0.008} &   1.182 & \tiny{ $-$0.020} & 168 & \tiny{  0.331} &   2.603 & \tiny{  0.530}\\
 34 & \tiny{ $-$0.025} &   0.631 & \tiny{ $-$0.029} &  79 & \tiny{ $-$0.025} &   0.772 & \tiny{ $-$0.031} & 124 & \tiny{ $-$0.007} &   1.198 & \tiny{ $-$0.019} & 169 & \tiny{  0.348} &   2.644 & \tiny{  0.571}\\
 35 & \tiny{ $-$0.025} &   0.632 & \tiny{ $-$0.029} &  80 & \tiny{ $-$0.025} &   0.777 & \tiny{ $-$0.031} & 125 & \tiny{ $-$0.006} &   1.215 & \tiny{ $-$0.017} & 170 & \tiny{  0.364} &   2.683 & \tiny{  0.612}\\
 36 & \tiny{ $-$0.025} &   0.634 & \tiny{ $-$0.029} &  81 & \tiny{ $-$0.025} &   0.782 & \tiny{ $-$0.031} & 126 & \tiny{ $-$0.004} &   1.232 & \tiny{ $-$0.016} & 171 & \tiny{  0.380} &   2.721 & \tiny{  0.654}\\
 37 & \tiny{ $-$0.025} &   0.636 & \tiny{ $-$0.029} &  82 & \tiny{ $-$0.025} &   0.788 & \tiny{ $-$0.031} & 127 & \tiny{ $-$0.003} &   1.249 & \tiny{ $-$0.015} & 172 & \tiny{  0.395} &   2.755 & \tiny{  0.694}\\
 38 & \tiny{ $-$0.025} &   0.637 & \tiny{ $-$0.029} &  83 & \tiny{ $-$0.025} &   0.793 & \tiny{ $-$0.031} & 128 & \tiny{ $-$0.001} &   1.267 & \tiny{ $-$0.013} & 173 & \tiny{  0.408} &   2.787 & \tiny{  0.734}\\
 39 & \tiny{ $-$0.025} &   0.639 & \tiny{ $-$0.029} &  84 & \tiny{ $-$0.025} &   0.799 & \tiny{ $-$0.031} & 129 & \tiny{  0.001} &   1.286 & \tiny{ $-$0.011} & 174 & \tiny{  0.421} &   2.816 & \tiny{  0.770}\\
 40 & \tiny{ $-$0.025} &   0.641 & \tiny{ $-$0.029} &  85 & \tiny{ $-$0.025} &   0.805 & \tiny{ $-$0.031} & 130 & \tiny{  0.003} &   1.305 & \tiny{ $-$0.009} & 175 & \tiny{  0.432} &   2.840 & \tiny{  0.804}\\
 41 & \tiny{ $-$0.025} &   0.643 & \tiny{ $-$0.029} &  86 & \tiny{ $-$0.025} &   0.811 & \tiny{ $-$0.031} & 131 & \tiny{  0.005} &   1.325 & \tiny{ $-$0.007} & 176 & \tiny{  0.442} &   2.861 & \tiny{  0.833}\\
 42 & \tiny{ $-$0.025} &   0.645 & \tiny{ $-$0.029} &  87 & \tiny{ $-$0.025} &   0.817 & \tiny{ $-$0.031} & 132 & \tiny{  0.007} &   1.345 & \tiny{ $-$0.005} & 177 & \tiny{  0.449} &   2.878 & \tiny{  0.857}\\
 43 & \tiny{ $-$0.025} &   0.647 & \tiny{ $-$0.029} &  88 & \tiny{ $-$0.025} &   0.823 & \tiny{ $-$0.031} & 133 & \tiny{  0.010} &   1.366 & \tiny{ $-$0.003} & 178 & \tiny{  0.455} &   2.890 & \tiny{  0.874}\\
 44 & \tiny{ $-$0.025} &   0.649 & \tiny{ $-$0.029} &  89 & \tiny{ $-$0.025} &   0.830 & \tiny{ $-$0.031} & 134 & \tiny{  0.012} &   1.388 & \tiny{ $-$0.000} & 179 & \tiny{  0.458} &   2.897 & \tiny{  0.885}\\
 45 & \tiny{ $-$0.025} &   0.652 & \tiny{ $-$0.029} &  90 & \tiny{ $-$0.025} &   0.836 & \tiny{ $-$0.031} & 135 & \tiny{  0.015} &   1.410 & \tiny{  0.003} & 180 & \tiny{  0.459} &   2.899 & \tiny{  0.889}\\
 \end{tabular}}
\caption{Epicyclic latitude correction factor for Mars. $\mu$ is in degrees. Note that $\bar{h}(360^\circ-\mu) = \bar{h}(\mu)$, and $\delta h_{\pm}(360^\circ-\mu) = \delta h_{\pm}(\mu)$. }\label{tlat2m}
\end{table}

\newpage
\begin{table}\centering
\small{ \begin{tabular}{crc}
$F (^\circ)$ & $\beta_0(^\circ)$ &
$F (^\circ)$ \\\hline
&&\\[-1.75ex]
000/180 &  0.000 & (180)/(360)\\
002/178 &  0.046 & (182)/(358)\\
004/176 &  0.091 & (184)/(356)\\
006/174 &  0.136 & (186)/(354)\\
008/172 &  0.182 & (188)/(352)\\
010/170 &  0.226 & (190)/(350)\\
012/168 &  0.271 & (192)/(348)\\
014/166 &  0.316 & (194)/(346)\\
016/164 &  0.360 & (196)/(344)\\
018/162 &  0.403 & (198)/(342)\\
020/160 &  0.446 & (200)/(340)\\
022/158 &  0.489 & (202)/(338)\\
024/156 &  0.531 & (204)/(336)\\
026/154 &  0.572 & (206)/(334)\\
028/152 &  0.612 & (208)/(332)\\
030/150 &  0.652 & (210)/(330)\\
032/148 &  0.691 & (212)/(328)\\
034/146 &  0.729 & (214)/(326)\\
036/144 &  0.767 & (216)/(324)\\
038/142 &  0.803 & (218)/(322)\\
040/140 &  0.838 & (220)/(320)\\
042/138 &  0.873 & (222)/(318)\\
044/136 &  0.906 & (224)/(316)\\
046/134 &  0.938 & (226)/(314)\\
048/132 &  0.969 & (228)/(312)\\
050/130 &  0.999 & (230)/(310)\\
052/128 &  1.028 & (232)/(308)\\
054/126 &  1.055 & (234)/(306)\\
056/124 &  1.081 & (236)/(304)\\
058/122 &  1.106 & (238)/(302)\\
060/120 &  1.130 & (240)/(300)\\
062/118 &  1.152 & (242)/(298)\\
064/116 &  1.172 & (244)/(296)\\
066/114 &  1.192 & (246)/(294)\\
068/112 &  1.209 & (248)/(292)\\
070/110 &  1.226 & (250)/(290)\\
072/108 &  1.240 & (252)/(288)\\
074/106 &  1.254 & (254)/(286)\\
076/104 &  1.266 & (256)/(284)\\
078/102 &  1.276 & (258)/(282)\\
080/100 &  1.284 & (260)/(280)\\
082/098 &  1.292 & (262)/(278)\\
084/096 &  1.297 & (264)/(276)\\
086/094 &  1.301 & (266)/(274)\\
088/092 &  1.303 & (268)/(272)\\
090/090 &  1.304 & (270)/(270)\\
\end{tabular}}
\caption{Deferential ecliptic latitude of Jupiter.  The latitude is minus the value shown
in the table if the argument is
in parenthesies. }\label{tlat1j}
\end{table}

\newpage
\begin{table}\centering
\small{ \begin{tabular}{rrrr|rrrr|rrrr|crrr}
$\mu$ & $\delta h_-$  & $\bar{h}~~~~$ & $\delta h_+$ &
$\mu$ & $\delta h_-$  & $\bar{h}~~~~$ & $\delta h_+$ &
$\mu$ & $\delta h_-$  & $\bar{h}~~~~$ & $\delta h_+$ &
$\mu$ & $\delta h_-$  & $\bar{h}~~~~$ & $\delta h_+$ \\\hline
&&&&&&&&&&&&&&&\\[-1.75ex]
  0 & \tiny{ $-$0.008} &   0.839 & \tiny{ $-$0.009} &  45 & \tiny{ $-$0.007} &   0.874 & \tiny{ $-$0.008} &  90 & \tiny{ $-$0.002} &   0.982 & \tiny{ $-$0.003} & 135 & \tiny{  0.009} &   1.143 & \tiny{  0.010}\\
  1 & \tiny{ $-$0.008} &   0.839 & \tiny{ $-$0.009} &  46 & \tiny{ $-$0.007} &   0.876 & \tiny{ $-$0.008} &  91 & \tiny{ $-$0.002} &   0.985 & \tiny{ $-$0.002} & 136 & \tiny{  0.009} &   1.147 & \tiny{  0.011}\\
  2 & \tiny{ $-$0.008} &   0.839 & \tiny{ $-$0.009} &  47 & \tiny{ $-$0.007} &   0.877 & \tiny{ $-$0.008} &  92 & \tiny{ $-$0.002} &   0.988 & \tiny{ $-$0.002} & 137 & \tiny{  0.010} &   1.150 & \tiny{  0.011}\\
  3 & \tiny{ $-$0.008} &   0.839 & \tiny{ $-$0.009} &  48 & \tiny{ $-$0.007} &   0.879 & \tiny{ $-$0.008} &  93 & \tiny{ $-$0.002} &   0.992 & \tiny{ $-$0.002} & 138 & \tiny{  0.010} &   1.154 & \tiny{  0.011}\\
  4 & \tiny{ $-$0.008} &   0.839 & \tiny{ $-$0.009} &  49 & \tiny{ $-$0.007} &   0.881 & \tiny{ $-$0.008} &  94 & \tiny{ $-$0.001} &   0.995 & \tiny{ $-$0.002} & 139 & \tiny{  0.010} &   1.157 & \tiny{  0.012}\\
  5 & \tiny{ $-$0.008} &   0.839 & \tiny{ $-$0.009} &  50 & \tiny{ $-$0.007} &   0.883 & \tiny{ $-$0.008} &  95 & \tiny{ $-$0.001} &   0.998 & \tiny{ $-$0.001} & 140 & \tiny{  0.010} &   1.160 & \tiny{  0.012}\\
  6 & \tiny{ $-$0.008} &   0.840 & \tiny{ $-$0.009} &  51 & \tiny{ $-$0.007} &   0.884 & \tiny{ $-$0.008} &  96 & \tiny{ $-$0.001} &   1.002 & \tiny{ $-$0.001} & 141 & \tiny{  0.011} &   1.164 & \tiny{  0.012}\\
  7 & \tiny{ $-$0.008} &   0.840 & \tiny{ $-$0.009} &  52 & \tiny{ $-$0.007} &   0.886 & \tiny{ $-$0.007} &  97 & \tiny{ $-$0.001} &   1.005 & \tiny{ $-$0.001} & 142 & \tiny{  0.011} &   1.167 & \tiny{  0.013}\\
  8 & \tiny{ $-$0.008} &   0.840 & \tiny{ $-$0.009} &  53 & \tiny{ $-$0.007} &   0.888 & \tiny{ $-$0.007} &  98 & \tiny{ $-$0.001} &   1.008 & \tiny{ $-$0.001} & 143 & \tiny{  0.011} &   1.170 & \tiny{  0.013}\\
  9 & \tiny{ $-$0.008} &   0.840 & \tiny{ $-$0.009} &  54 & \tiny{ $-$0.006} &   0.890 & \tiny{ $-$0.007} &  99 & \tiny{ $-$0.000} &   1.012 & \tiny{ $-$0.001} & 144 & \tiny{  0.012} &   1.173 & \tiny{  0.013}\\
 10 & \tiny{ $-$0.008} &   0.841 & \tiny{ $-$0.009} &  55 & \tiny{ $-$0.006} &   0.892 & \tiny{ $-$0.007} & 100 & \tiny{ $-$0.000} &   1.015 & \tiny{ $-$0.000} & 145 & \tiny{  0.012} &   1.177 & \tiny{  0.014}\\
 11 & \tiny{ $-$0.008} &   0.841 & \tiny{ $-$0.009} &  56 & \tiny{ $-$0.006} &   0.894 & \tiny{ $-$0.007} & 101 & \tiny{  0.000} &   1.019 & \tiny{ $-$0.000} & 146 & \tiny{  0.012} &   1.180 & \tiny{  0.014}\\
 12 & \tiny{ $-$0.008} &   0.841 & \tiny{ $-$0.009} &  57 & \tiny{ $-$0.006} &   0.896 & \tiny{ $-$0.007} & 102 & \tiny{  0.000} &   1.022 & \tiny{  0.000} & 147 & \tiny{  0.012} &   1.183 & \tiny{  0.014}\\
 13 & \tiny{ $-$0.008} &   0.842 & \tiny{ $-$0.009} &  58 & \tiny{ $-$0.006} &   0.898 & \tiny{ $-$0.007} & 103 & \tiny{  0.000} &   1.026 & \tiny{  0.000} & 148 & \tiny{  0.013} &   1.186 & \tiny{  0.015}\\
 14 & \tiny{ $-$0.008} &   0.842 & \tiny{ $-$0.009} &  59 & \tiny{ $-$0.006} &   0.900 & \tiny{ $-$0.007} & 104 & \tiny{  0.001} &   1.029 & \tiny{  0.001} & 149 & \tiny{  0.013} &   1.189 & \tiny{  0.015}\\
 15 & \tiny{ $-$0.008} &   0.843 & \tiny{ $-$0.009} &  60 & \tiny{ $-$0.006} &   0.902 & \tiny{ $-$0.007} & 105 & \tiny{  0.001} &   1.033 & \tiny{  0.001} & 150 & \tiny{  0.013} &   1.192 & \tiny{  0.015}\\
 16 & \tiny{ $-$0.008} &   0.843 & \tiny{ $-$0.009} &  61 & \tiny{ $-$0.006} &   0.904 & \tiny{ $-$0.007} & 106 & \tiny{  0.001} &   1.036 & \tiny{  0.001} & 151 & \tiny{  0.014} &   1.194 & \tiny{  0.016}\\
 17 & \tiny{ $-$0.008} &   0.844 & \tiny{ $-$0.009} &  62 & \tiny{ $-$0.006} &   0.906 & \tiny{ $-$0.007} & 107 & \tiny{  0.001} &   1.040 & \tiny{  0.001} & 152 & \tiny{  0.014} &   1.197 & \tiny{  0.016}\\
 18 & \tiny{ $-$0.008} &   0.845 & \tiny{ $-$0.009} &  63 & \tiny{ $-$0.006} &   0.909 & \tiny{ $-$0.006} & 108 & \tiny{  0.002} &   1.044 & \tiny{  0.002} & 153 & \tiny{  0.014} &   1.200 & \tiny{  0.016}\\
 19 & \tiny{ $-$0.008} &   0.845 & \tiny{ $-$0.009} &  64 & \tiny{ $-$0.006} &   0.911 & \tiny{ $-$0.006} & 109 & \tiny{  0.002} &   1.047 & \tiny{  0.002} & 154 & \tiny{  0.014} &   1.202 & \tiny{  0.017}\\
 20 & \tiny{ $-$0.008} &   0.846 & \tiny{ $-$0.009} &  65 & \tiny{ $-$0.005} &   0.913 & \tiny{ $-$0.006} & 110 & \tiny{  0.002} &   1.051 & \tiny{  0.002} & 155 & \tiny{  0.015} &   1.205 & \tiny{  0.017}\\
 21 & \tiny{ $-$0.008} &   0.847 & \tiny{ $-$0.009} &  66 & \tiny{ $-$0.005} &   0.916 & \tiny{ $-$0.006} & 111 & \tiny{  0.002} &   1.055 & \tiny{  0.003} & 156 & \tiny{  0.015} &   1.207 & \tiny{  0.017}\\
 22 & \tiny{ $-$0.008} &   0.847 & \tiny{ $-$0.009} &  67 & \tiny{ $-$0.005} &   0.918 & \tiny{ $-$0.006} & 112 & \tiny{  0.003} &   1.058 & \tiny{  0.003} & 157 & \tiny{  0.015} &   1.210 & \tiny{  0.017}\\
 23 & \tiny{ $-$0.008} &   0.848 & \tiny{ $-$0.009} &  68 & \tiny{ $-$0.005} &   0.920 & \tiny{ $-$0.006} & 113 & \tiny{  0.003} &   1.062 & \tiny{  0.003} & 158 & \tiny{  0.015} &   1.212 & \tiny{  0.018}\\
 24 & \tiny{ $-$0.008} &   0.849 & \tiny{ $-$0.009} &  69 & \tiny{ $-$0.005} &   0.923 & \tiny{ $-$0.006} & 114 & \tiny{  0.003} &   1.066 & \tiny{  0.003} & 159 & \tiny{  0.015} &   1.214 & \tiny{  0.018}\\
 25 & \tiny{ $-$0.008} &   0.850 & \tiny{ $-$0.009} &  70 & \tiny{ $-$0.005} &   0.925 & \tiny{ $-$0.006} & 115 & \tiny{  0.003} &   1.069 & \tiny{  0.004} & 160 & \tiny{  0.016} &   1.216 & \tiny{  0.018}\\
 26 & \tiny{ $-$0.008} &   0.851 & \tiny{ $-$0.009} &  71 & \tiny{ $-$0.005} &   0.928 & \tiny{ $-$0.006} & 116 & \tiny{  0.004} &   1.073 & \tiny{  0.004} & 161 & \tiny{  0.016} &   1.218 & \tiny{  0.018}\\
 27 & \tiny{ $-$0.008} &   0.852 & \tiny{ $-$0.009} &  72 & \tiny{ $-$0.005} &   0.930 & \tiny{ $-$0.005} & 117 & \tiny{  0.004} &   1.077 & \tiny{  0.004} & 162 & \tiny{  0.016} &   1.220 & \tiny{  0.019}\\
 28 & \tiny{ $-$0.008} &   0.853 & \tiny{ $-$0.009} &  73 & \tiny{ $-$0.005} &   0.933 & \tiny{ $-$0.005} & 118 & \tiny{  0.004} &   1.080 & \tiny{  0.005} & 163 & \tiny{  0.016} &   1.222 & \tiny{  0.019}\\
 29 & \tiny{ $-$0.008} &   0.854 & \tiny{ $-$0.009} &  74 & \tiny{ $-$0.004} &   0.935 & \tiny{ $-$0.005} & 119 & \tiny{  0.004} &   1.084 & \tiny{  0.005} & 164 & \tiny{  0.016} &   1.224 & \tiny{  0.019}\\
 30 & \tiny{ $-$0.008} &   0.855 & \tiny{ $-$0.009} &  75 & \tiny{ $-$0.004} &   0.938 & \tiny{ $-$0.005} & 120 & \tiny{  0.005} &   1.088 & \tiny{  0.005} & 165 & \tiny{  0.016} &   1.225 & \tiny{  0.019}\\
 31 & \tiny{ $-$0.008} &   0.856 & \tiny{ $-$0.009} &  76 & \tiny{ $-$0.004} &   0.941 & \tiny{ $-$0.005} & 121 & \tiny{  0.005} &   1.092 & \tiny{  0.006} & 166 & \tiny{  0.017} &   1.227 & \tiny{  0.019}\\
 32 & \tiny{ $-$0.008} &   0.857 & \tiny{ $-$0.009} &  77 & \tiny{ $-$0.004} &   0.944 & \tiny{ $-$0.005} & 122 & \tiny{  0.005} &   1.095 & \tiny{  0.006} & 167 & \tiny{  0.017} &   1.228 & \tiny{  0.020}\\
 33 & \tiny{ $-$0.008} &   0.858 & \tiny{ $-$0.009} &  78 & \tiny{ $-$0.004} &   0.946 & \tiny{ $-$0.005} & 123 & \tiny{  0.006} &   1.099 & \tiny{  0.006} & 168 & \tiny{  0.017} &   1.230 & \tiny{  0.020}\\
 34 & \tiny{ $-$0.008} &   0.859 & \tiny{ $-$0.009} &  79 & \tiny{ $-$0.004} &   0.949 & \tiny{ $-$0.004} & 124 & \tiny{  0.006} &   1.103 & \tiny{  0.007} & 169 & \tiny{  0.017} &   1.231 & \tiny{  0.020}\\
 35 & \tiny{ $-$0.008} &   0.860 & \tiny{ $-$0.009} &  80 & \tiny{ $-$0.004} &   0.952 & \tiny{ $-$0.004} & 125 & \tiny{  0.006} &   1.107 & \tiny{  0.007} & 170 & \tiny{  0.017} &   1.232 & \tiny{  0.020}\\
 36 & \tiny{ $-$0.008} &   0.861 & \tiny{ $-$0.008} &  81 & \tiny{ $-$0.004} &   0.955 & \tiny{ $-$0.004} & 126 & \tiny{  0.006} &   1.110 & \tiny{  0.007} & 171 & \tiny{  0.017} &   1.233 & \tiny{  0.020}\\
 37 & \tiny{ $-$0.008} &   0.863 & \tiny{ $-$0.008} &  82 & \tiny{ $-$0.003} &   0.958 & \tiny{ $-$0.004} & 127 & \tiny{  0.007} &   1.114 & \tiny{  0.008} & 172 & \tiny{  0.017} &   1.234 & \tiny{  0.020}\\
 38 & \tiny{ $-$0.007} &   0.864 & \tiny{ $-$0.008} &  83 & \tiny{ $-$0.003} &   0.961 & \tiny{ $-$0.004} & 128 & \tiny{  0.007} &   1.118 & \tiny{  0.008} & 173 & \tiny{  0.017} &   1.235 & \tiny{  0.020}\\
 39 & \tiny{ $-$0.007} &   0.865 & \tiny{ $-$0.008} &  84 & \tiny{ $-$0.003} &   0.964 & \tiny{ $-$0.004} & 129 & \tiny{  0.007} &   1.121 & \tiny{  0.008} & 174 & \tiny{  0.018} &   1.236 & \tiny{  0.021}\\
 40 & \tiny{ $-$0.007} &   0.867 & \tiny{ $-$0.008} &  85 & \tiny{ $-$0.003} &   0.967 & \tiny{ $-$0.003} & 130 & \tiny{  0.008} &   1.125 & \tiny{  0.009} & 175 & \tiny{  0.018} &   1.236 & \tiny{  0.021}\\
 41 & \tiny{ $-$0.007} &   0.868 & \tiny{ $-$0.008} &  86 & \tiny{ $-$0.003} &   0.970 & \tiny{ $-$0.003} & 131 & \tiny{  0.008} &   1.129 & \tiny{  0.009} & 176 & \tiny{  0.018} &   1.237 & \tiny{  0.021}\\
 42 & \tiny{ $-$0.007} &   0.870 & \tiny{ $-$0.008} &  87 & \tiny{ $-$0.003} &   0.973 & \tiny{ $-$0.003} & 132 & \tiny{  0.008} &   1.132 & \tiny{  0.009} & 177 & \tiny{  0.018} &   1.237 & \tiny{  0.021}\\
 43 & \tiny{ $-$0.007} &   0.871 & \tiny{ $-$0.008} &  88 & \tiny{ $-$0.002} &   0.976 & \tiny{ $-$0.003} & 133 & \tiny{  0.008} &   1.136 & \tiny{  0.010} & 178 & \tiny{  0.018} &   1.237 & \tiny{  0.021}\\
 44 & \tiny{ $-$0.007} &   0.873 & \tiny{ $-$0.008} &  89 & \tiny{ $-$0.002} &   0.979 & \tiny{ $-$0.003} & 134 & \tiny{  0.009} &   1.140 & \tiny{  0.010} & 179 & \tiny{  0.018} &   1.238 & \tiny{  0.021}\\
 45 & \tiny{ $-$0.007} &   0.874 & \tiny{ $-$0.008} &  90 & \tiny{ $-$0.002} &   0.982 & \tiny{ $-$0.003} & 135 & \tiny{  0.009} &   1.143 & \tiny{  0.010} & 180 & \tiny{  0.018} &   1.238 & \tiny{  0.021}\\
 \end{tabular}}
\caption{Epicyclic latitude correction factor for Jupiter. $\mu$ is in degrees. Note that $\bar{h}(360^\circ-\mu) = \bar{h}(\mu)$, and $\delta h_{\pm}(360^\circ-\mu) = \delta h_{\pm}(\mu)$. }\label{tlat2j}
\end{table}

\newpage
\begin{table}\centering
\small{ \begin{tabular}{crc}
$F (^\circ)$ & $\beta_0(^\circ)$ &
$F (^\circ)$ \\\hline
&&\\[-1.75ex]
000/180 &  0.000 & (180)/(360)\\
002/178 &  0.087 & (182)/(358)\\
004/176 &  0.173 & (184)/(356)\\
006/174 &  0.260 & (186)/(354)\\
008/172 &  0.346 & (188)/(352)\\
010/170 &  0.432 & (190)/(350)\\
012/168 &  0.517 & (192)/(348)\\
014/166 &  0.601 & (194)/(346)\\
016/164 &  0.685 & (196)/(344)\\
018/162 &  0.768 & (198)/(342)\\
020/160 &  0.850 & (200)/(340)\\
022/158 &  0.931 & (202)/(338)\\
024/156 &  1.011 & (204)/(336)\\
026/154 &  1.089 & (206)/(334)\\
028/152 &  1.167 & (208)/(332)\\
030/150 &  1.243 & (210)/(330)\\
032/148 &  1.317 & (212)/(328)\\
034/146 &  1.390 & (214)/(326)\\
036/144 &  1.461 & (216)/(324)\\
038/142 &  1.530 & (218)/(322)\\
040/140 &  1.597 & (220)/(320)\\
042/138 &  1.663 & (222)/(318)\\
044/136 &  1.726 & (224)/(316)\\
046/134 &  1.788 & (226)/(314)\\
048/132 &  1.847 & (228)/(312)\\
050/130 &  1.904 & (230)/(310)\\
052/128 &  1.958 & (232)/(308)\\
054/126 &  2.011 & (234)/(306)\\
056/124 &  2.060 & (236)/(304)\\
058/122 &  2.108 & (238)/(302)\\
060/120 &  2.152 & (240)/(300)\\
062/118 &  2.194 & (242)/(298)\\
064/116 &  2.234 & (244)/(296)\\
066/114 &  2.270 & (246)/(294)\\
068/112 &  2.304 & (248)/(292)\\
070/110 &  2.335 & (250)/(290)\\
072/108 &  2.364 & (252)/(288)\\
074/106 &  2.389 & (254)/(286)\\
076/104 &  2.411 & (256)/(284)\\
078/102 &  2.431 & (258)/(282)\\
080/100 &  2.447 & (260)/(280)\\
082/098 &  2.461 & (262)/(278)\\
084/096 &  2.472 & (264)/(276)\\
086/094 &  2.479 & (266)/(274)\\
088/092 &  2.484 & (268)/(272)\\
090/090 &  2.485 & (270)/(270)\\
\end{tabular}}
\caption{Deferential ecliptic latitude of Saturn.  The latitude is minus the value shown
in the table if the argument is
in parenthesies. }\label{tlat1s}
\end{table}

\newpage
\begin{table}\centering
\small{ \begin{tabular}{rrrr|rrrr|rrrr|crrr}
$\mu$ & $\delta h_-$  & $\bar{h}~~~~$ & $\delta h_+$ &
$\mu$ & $\delta h_-$  & $\bar{h}~~~~$ & $\delta h_+$ &
$\mu$ & $\delta h_-$  & $\bar{h}~~~~$ & $\delta h_+$ &
$\mu$ & $\delta h_-$  & $\bar{h}~~~~$ & $\delta h_+$ \\\hline
&&&&&&&&&&&&&&&\\[-1.75ex]
  0 & \tiny{ $-$0.006} &   0.905 & \tiny{ $-$0.006} &  45 & \tiny{ $-$0.005} &   0.929 & \tiny{ $-$0.005} &  90 & \tiny{ $-$0.001} &   0.995 & \tiny{ $-$0.001} & 135 & \tiny{  0.005} &   1.077 & \tiny{  0.006}\\
  1 & \tiny{ $-$0.006} &   0.905 & \tiny{ $-$0.006} &  46 & \tiny{ $-$0.004} &   0.930 & \tiny{ $-$0.005} &  91 & \tiny{ $-$0.001} &   0.996 & \tiny{ $-$0.001} & 136 & \tiny{  0.005} &   1.078 & \tiny{  0.006}\\
  2 & \tiny{ $-$0.006} &   0.905 & \tiny{ $-$0.006} &  47 & \tiny{ $-$0.004} &   0.931 & \tiny{ $-$0.005} &  92 & \tiny{ $-$0.000} &   0.998 & \tiny{ $-$0.001} & 137 & \tiny{  0.005} &   1.080 & \tiny{  0.006}\\
  3 & \tiny{ $-$0.006} &   0.905 & \tiny{ $-$0.006} &  48 & \tiny{ $-$0.004} &   0.932 & \tiny{ $-$0.005} &  93 & \tiny{ $-$0.000} &   1.000 & \tiny{ $-$0.000} & 138 & \tiny{  0.006} &   1.081 & \tiny{  0.006}\\
  4 & \tiny{ $-$0.006} &   0.905 & \tiny{ $-$0.006} &  49 & \tiny{ $-$0.004} &   0.933 & \tiny{ $-$0.005} &  94 & \tiny{ $-$0.000} &   1.002 & \tiny{ $-$0.000} & 139 & \tiny{  0.006} &   1.083 & \tiny{  0.007}\\
  5 & \tiny{ $-$0.006} &   0.905 & \tiny{ $-$0.006} &  50 & \tiny{ $-$0.004} &   0.934 & \tiny{ $-$0.005} &  95 & \tiny{ $-$0.000} &   1.004 & \tiny{ $-$0.000} & 140 & \tiny{  0.006} &   1.084 & \tiny{  0.007}\\
  6 & \tiny{ $-$0.006} &   0.906 & \tiny{ $-$0.006} &  51 & \tiny{ $-$0.004} &   0.935 & \tiny{ $-$0.005} &  96 & \tiny{  0.000} &   1.006 & \tiny{ $-$0.000} & 141 & \tiny{  0.006} &   1.086 & \tiny{  0.007}\\
  7 & \tiny{ $-$0.006} &   0.906 & \tiny{ $-$0.006} &  52 & \tiny{ $-$0.004} &   0.937 & \tiny{ $-$0.005} &  97 & \tiny{  0.000} &   1.007 & \tiny{  0.000} & 142 & \tiny{  0.006} &   1.087 & \tiny{  0.007}\\
  8 & \tiny{ $-$0.006} &   0.906 & \tiny{ $-$0.006} &  53 & \tiny{ $-$0.004} &   0.938 & \tiny{ $-$0.005} &  98 & \tiny{  0.000} &   1.009 & \tiny{  0.000} & 143 & \tiny{  0.006} &   1.089 & \tiny{  0.007}\\
  9 & \tiny{ $-$0.006} &   0.906 & \tiny{ $-$0.006} &  54 & \tiny{ $-$0.004} &   0.939 & \tiny{ $-$0.005} &  99 & \tiny{  0.000} &   1.011 & \tiny{  0.000} & 144 & \tiny{  0.006} &   1.090 & \tiny{  0.007}\\
 10 & \tiny{ $-$0.006} &   0.906 & \tiny{ $-$0.006} &  55 & \tiny{ $-$0.004} &   0.940 & \tiny{ $-$0.004} & 100 & \tiny{  0.001} &   1.013 & \tiny{  0.001} & 145 & \tiny{  0.006} &   1.091 & \tiny{  0.007}\\
 11 & \tiny{ $-$0.006} &   0.907 & \tiny{ $-$0.006} &  56 & \tiny{ $-$0.004} &   0.942 & \tiny{ $-$0.004} & 101 & \tiny{  0.001} &   1.015 & \tiny{  0.001} & 146 & \tiny{  0.006} &   1.093 & \tiny{  0.008}\\
 12 & \tiny{ $-$0.006} &   0.907 & \tiny{ $-$0.006} &  57 & \tiny{ $-$0.004} &   0.943 & \tiny{ $-$0.004} & 102 & \tiny{  0.001} &   1.017 & \tiny{  0.001} & 147 & \tiny{  0.007} &   1.094 & \tiny{  0.008}\\
 13 & \tiny{ $-$0.006} &   0.907 & \tiny{ $-$0.006} &  58 & \tiny{ $-$0.004} &   0.944 & \tiny{ $-$0.004} & 103 & \tiny{  0.001} &   1.019 & \tiny{  0.001} & 148 & \tiny{  0.007} &   1.095 & \tiny{  0.008}\\
 14 & \tiny{ $-$0.006} &   0.908 & \tiny{ $-$0.006} &  59 & \tiny{ $-$0.004} &   0.945 & \tiny{ $-$0.004} & 104 & \tiny{  0.001} &   1.020 & \tiny{  0.001} & 149 & \tiny{  0.007} &   1.097 & \tiny{  0.008}\\
 15 & \tiny{ $-$0.006} &   0.908 & \tiny{ $-$0.006} &  60 & \tiny{ $-$0.004} &   0.947 & \tiny{ $-$0.004} & 105 & \tiny{  0.001} &   1.022 & \tiny{  0.001} & 150 & \tiny{  0.007} &   1.098 & \tiny{  0.008}\\
 16 & \tiny{ $-$0.006} &   0.908 & \tiny{ $-$0.006} &  61 & \tiny{ $-$0.003} &   0.948 & \tiny{ $-$0.004} & 106 & \tiny{  0.001} &   1.024 & \tiny{  0.001} & 151 & \tiny{  0.007} &   1.099 & \tiny{  0.008}\\
 17 & \tiny{ $-$0.006} &   0.909 & \tiny{ $-$0.006} &  62 & \tiny{ $-$0.003} &   0.949 & \tiny{ $-$0.004} & 107 & \tiny{  0.001} &   1.026 & \tiny{  0.002} & 152 & \tiny{  0.007} &   1.100 & \tiny{  0.008}\\
 18 & \tiny{ $-$0.006} &   0.909 & \tiny{ $-$0.006} &  63 & \tiny{ $-$0.003} &   0.951 & \tiny{ $-$0.004} & 108 & \tiny{  0.002} &   1.028 & \tiny{  0.002} & 153 & \tiny{  0.007} &   1.101 & \tiny{  0.008}\\
 19 & \tiny{ $-$0.005} &   0.909 & \tiny{ $-$0.006} &  64 & \tiny{ $-$0.003} &   0.952 & \tiny{ $-$0.004} & 109 & \tiny{  0.002} &   1.030 & \tiny{  0.002} & 154 & \tiny{  0.007} &   1.103 & \tiny{  0.009}\\
 20 & \tiny{ $-$0.005} &   0.910 & \tiny{ $-$0.006} &  65 & \tiny{ $-$0.003} &   0.954 & \tiny{ $-$0.004} & 110 & \tiny{  0.002} &   1.032 & \tiny{  0.002} & 155 & \tiny{  0.007} &   1.104 & \tiny{  0.009}\\
 21 & \tiny{ $-$0.005} &   0.910 & \tiny{ $-$0.006} &  66 & \tiny{ $-$0.003} &   0.955 & \tiny{ $-$0.004} & 111 & \tiny{  0.002} &   1.034 & \tiny{  0.002} & 156 & \tiny{  0.007} &   1.105 & \tiny{  0.009}\\
 22 & \tiny{ $-$0.005} &   0.911 & \tiny{ $-$0.006} &  67 & \tiny{ $-$0.003} &   0.957 & \tiny{ $-$0.003} & 112 & \tiny{  0.002} &   1.036 & \tiny{  0.002} & 157 & \tiny{  0.008} &   1.106 & \tiny{  0.009}\\
 23 & \tiny{ $-$0.005} &   0.911 & \tiny{ $-$0.006} &  68 & \tiny{ $-$0.003} &   0.958 & \tiny{ $-$0.003} & 113 & \tiny{  0.002} &   1.037 & \tiny{  0.003} & 158 & \tiny{  0.008} &   1.107 & \tiny{  0.009}\\
 24 & \tiny{ $-$0.005} &   0.912 & \tiny{ $-$0.006} &  69 & \tiny{ $-$0.003} &   0.960 & \tiny{ $-$0.003} & 114 & \tiny{  0.002} &   1.039 & \tiny{  0.003} & 159 & \tiny{  0.008} &   1.107 & \tiny{  0.009}\\
 25 & \tiny{ $-$0.005} &   0.913 & \tiny{ $-$0.006} &  70 & \tiny{ $-$0.003} &   0.961 & \tiny{ $-$0.003} & 115 & \tiny{  0.002} &   1.041 & \tiny{  0.003} & 160 & \tiny{  0.008} &   1.108 & \tiny{  0.009}\\
 26 & \tiny{ $-$0.005} &   0.913 & \tiny{ $-$0.006} &  71 & \tiny{ $-$0.003} &   0.963 & \tiny{ $-$0.003} & 116 & \tiny{  0.003} &   1.043 & \tiny{  0.003} & 161 & \tiny{  0.008} &   1.109 & \tiny{  0.009}\\
 27 & \tiny{ $-$0.005} &   0.914 & \tiny{ $-$0.006} &  72 & \tiny{ $-$0.003} &   0.964 & \tiny{ $-$0.003} & 117 & \tiny{  0.003} &   1.045 & \tiny{  0.003} & 162 & \tiny{  0.008} &   1.110 & \tiny{  0.009}\\
 28 & \tiny{ $-$0.005} &   0.914 & \tiny{ $-$0.006} &  73 & \tiny{ $-$0.002} &   0.966 & \tiny{ $-$0.003} & 118 & \tiny{  0.003} &   1.047 & \tiny{  0.003} & 163 & \tiny{  0.008} &   1.111 & \tiny{  0.009}\\
 29 & \tiny{ $-$0.005} &   0.915 & \tiny{ $-$0.006} &  74 & \tiny{ $-$0.002} &   0.967 & \tiny{ $-$0.003} & 119 & \tiny{  0.003} &   1.049 & \tiny{  0.003} & 164 & \tiny{  0.008} &   1.111 & \tiny{  0.009}\\
 30 & \tiny{ $-$0.005} &   0.916 & \tiny{ $-$0.006} &  75 & \tiny{ $-$0.002} &   0.969 & \tiny{ $-$0.003} & 120 & \tiny{  0.003} &   1.050 & \tiny{  0.004} & 165 & \tiny{  0.008} &   1.112 & \tiny{  0.009}\\
 31 & \tiny{ $-$0.005} &   0.916 & \tiny{ $-$0.006} &  76 & \tiny{ $-$0.002} &   0.971 & \tiny{ $-$0.003} & 121 & \tiny{  0.003} &   1.052 & \tiny{  0.004} & 166 & \tiny{  0.008} &   1.113 & \tiny{  0.010}\\
 32 & \tiny{ $-$0.005} &   0.917 & \tiny{ $-$0.006} &  77 & \tiny{ $-$0.002} &   0.972 & \tiny{ $-$0.002} & 122 & \tiny{  0.003} &   1.054 & \tiny{  0.004} & 167 & \tiny{  0.008} &   1.113 & \tiny{  0.010}\\
 33 & \tiny{ $-$0.005} &   0.918 & \tiny{ $-$0.006} &  78 & \tiny{ $-$0.002} &   0.974 & \tiny{ $-$0.002} & 123 & \tiny{  0.004} &   1.056 & \tiny{  0.004} & 168 & \tiny{  0.008} &   1.114 & \tiny{  0.010}\\
 34 & \tiny{ $-$0.005} &   0.919 & \tiny{ $-$0.006} &  79 & \tiny{ $-$0.002} &   0.975 & \tiny{ $-$0.002} & 124 & \tiny{  0.004} &   1.058 & \tiny{  0.004} & 169 & \tiny{  0.008} &   1.114 & \tiny{  0.010}\\
 35 & \tiny{ $-$0.005} &   0.920 & \tiny{ $-$0.006} &  80 & \tiny{ $-$0.002} &   0.977 & \tiny{ $-$0.002} & 125 & \tiny{  0.004} &   1.060 & \tiny{  0.004} & 170 & \tiny{  0.008} &   1.115 & \tiny{  0.010}\\
 36 & \tiny{ $-$0.005} &   0.920 & \tiny{ $-$0.006} &  81 & \tiny{ $-$0.002} &   0.979 & \tiny{ $-$0.002} & 126 & \tiny{  0.004} &   1.061 & \tiny{  0.005} & 171 & \tiny{  0.008} &   1.115 & \tiny{  0.010}\\
 37 & \tiny{ $-$0.005} &   0.921 & \tiny{ $-$0.006} &  82 & \tiny{ $-$0.002} &   0.981 & \tiny{ $-$0.002} & 127 & \tiny{  0.004} &   1.063 & \tiny{  0.005} & 172 & \tiny{  0.008} &   1.116 & \tiny{  0.010}\\
 38 & \tiny{ $-$0.005} &   0.922 & \tiny{ $-$0.006} &  83 & \tiny{ $-$0.001} &   0.982 & \tiny{ $-$0.002} & 128 & \tiny{  0.004} &   1.065 & \tiny{  0.005} & 173 & \tiny{  0.008} &   1.116 & \tiny{  0.010}\\
 39 & \tiny{ $-$0.005} &   0.923 & \tiny{ $-$0.005} &  84 & \tiny{ $-$0.001} &   0.984 & \tiny{ $-$0.002} & 129 & \tiny{  0.004} &   1.067 & \tiny{  0.005} & 174 & \tiny{  0.008} &   1.116 & \tiny{  0.010}\\
 40 & \tiny{ $-$0.005} &   0.924 & \tiny{ $-$0.005} &  85 & \tiny{ $-$0.001} &   0.986 & \tiny{ $-$0.001} & 130 & \tiny{  0.005} &   1.068 & \tiny{  0.005} & 175 & \tiny{  0.008} &   1.116 & \tiny{  0.010}\\
 41 & \tiny{ $-$0.005} &   0.925 & \tiny{ $-$0.005} &  86 & \tiny{ $-$0.001} &   0.987 & \tiny{ $-$0.001} & 131 & \tiny{  0.005} &   1.070 & \tiny{  0.005} & 176 & \tiny{  0.009} &   1.117 & \tiny{  0.010}\\
 42 & \tiny{ $-$0.005} &   0.926 & \tiny{ $-$0.005} &  87 & \tiny{ $-$0.001} &   0.989 & \tiny{ $-$0.001} & 132 & \tiny{  0.005} &   1.072 & \tiny{  0.006} & 177 & \tiny{  0.009} &   1.117 & \tiny{  0.010}\\
 43 & \tiny{ $-$0.005} &   0.927 & \tiny{ $-$0.005} &  88 & \tiny{ $-$0.001} &   0.991 & \tiny{ $-$0.001} & 133 & \tiny{  0.005} &   1.073 & \tiny{  0.006} & 178 & \tiny{  0.009} &   1.117 & \tiny{  0.010}\\
 44 & \tiny{ $-$0.005} &   0.928 & \tiny{ $-$0.005} &  89 & \tiny{ $-$0.001} &   0.993 & \tiny{ $-$0.001} & 134 & \tiny{  0.005} &   1.075 & \tiny{  0.006} & 179 & \tiny{  0.009} &   1.117 & \tiny{  0.010}\\
 45 & \tiny{ $-$0.005} &   0.929 & \tiny{ $-$0.005} &  90 & \tiny{ $-$0.001} &   0.995 & \tiny{ $-$0.001} & 135 & \tiny{  0.005} &   1.077 & \tiny{  0.006} & 180 & \tiny{  0.009} &   1.117 & \tiny{  0.010}\\
\end{tabular}}
\caption{Epicyclic latitude correction factor for Saturn. $\mu$ is in degrees. Note that $\bar{h}(360^\circ-\mu) = \bar{h}(\mu)$, and $\delta h_{\pm}(360^\circ-\mu) = \delta h_{\pm}(\mu)$. }\label{tlat2s}
\end{table}

\newpage
\begin{table}\centering
\small{ \begin{tabular}{crc}
$F (^\circ)$ & $\beta_0(^\circ)$ &
$F (^\circ)$ \\\hline
&&\\[-1.75ex]
000/180 &  0.000 & (180)/(360)\\
002/178 &  0.085 & (182)/(358)\\
004/176 &  0.170 & (184)/(356)\\
006/174 &  0.255 & (186)/(354)\\
008/172 &  0.339 & (188)/(352)\\
010/170 &  0.423 & (190)/(350)\\
012/168 &  0.506 & (192)/(348)\\
014/166 &  0.589 & (194)/(346)\\
016/164 &  0.671 & (196)/(344)\\
018/162 &  0.753 & (198)/(342)\\
020/160 &  0.833 & (200)/(340)\\
022/158 &  0.912 & (202)/(338)\\
024/156 &  0.991 & (204)/(336)\\
026/154 &  1.068 & (206)/(334)\\
028/152 &  1.143 & (208)/(332)\\
030/150 &  1.218 & (210)/(330)\\
032/148 &  1.291 & (212)/(328)\\
034/146 &  1.362 & (214)/(326)\\
036/144 &  1.432 & (216)/(324)\\
038/142 &  1.500 & (218)/(322)\\
040/140 &  1.566 & (220)/(320)\\
042/138 &  1.630 & (222)/(318)\\
044/136 &  1.692 & (224)/(316)\\
046/134 &  1.752 & (226)/(314)\\
048/132 &  1.810 & (228)/(312)\\
050/130 &  1.866 & (230)/(310)\\
052/128 &  1.919 & (232)/(308)\\
054/126 &  1.970 & (234)/(306)\\
056/124 &  2.019 & (236)/(304)\\
058/122 &  2.066 & (238)/(302)\\
060/120 &  2.109 & (240)/(300)\\
062/118 &  2.151 & (242)/(298)\\
064/116 &  2.189 & (244)/(296)\\
066/114 &  2.225 & (246)/(294)\\
068/112 &  2.258 & (248)/(292)\\
070/110 &  2.289 & (250)/(290)\\
072/108 &  2.316 & (252)/(288)\\
074/106 &  2.341 & (254)/(286)\\
076/104 &  2.363 & (256)/(284)\\
078/102 &  2.382 & (258)/(282)\\
080/100 &  2.399 & (260)/(280)\\
082/098 &  2.412 & (262)/(278)\\
084/096 &  2.422 & (264)/(276)\\
086/094 &  2.430 & (266)/(274)\\
088/092 &  2.434 & (268)/(272)\\
090/090 &  2.436 & (270)/(270)\\
\end{tabular}}
\caption{\em Epicyclic ecliptic latitude of Venus.  The latitude is minus the value shown
in the table if the argument is
in parenthesies. }\label{tlat1v}
\end{table}

\newpage
\begin{table}\centering
\small{ \begin{tabular}{rrrr|rrrr|rrrr|crrr}
$\mu$ & $\delta h_-$  & $\bar{h}~~~~$ & $\delta h_+$ &
$\mu$ & $\delta h_-$  & $\bar{h}~~~~$ & $\delta h_+$ &
$\mu$ & $\delta h_-$  & $\bar{h}~~~~$ & $\delta h_+$ &
$\mu$ & $\delta h_-$  & $\bar{h}~~~~$ & $\delta h_+$ \\\hline
&&&&&&&&&&&&&&&\\[-1.75ex]
  0 & \tiny{  0.008} &   0.580 & \tiny{  0.008} &  45 & \tiny{  0.009} &   0.627 & \tiny{  0.009} &  90 & \tiny{  0.012} &   0.810 & \tiny{  0.013} & 135 & \tiny{  0.032} &   1.413 & \tiny{  0.033}\\
  1 & \tiny{  0.008} &   0.580 & \tiny{  0.008} &  46 & \tiny{  0.009} &   0.629 & \tiny{  0.009} &  91 & \tiny{  0.013} &   0.817 & \tiny{  0.013} & 136 & \tiny{  0.033} &   1.439 & \tiny{  0.034}\\
  2 & \tiny{  0.008} &   0.580 & \tiny{  0.008} &  47 & \tiny{  0.009} &   0.631 & \tiny{  0.009} &  92 & \tiny{  0.013} &   0.824 & \tiny{  0.013} & 137 & \tiny{  0.034} &   1.466 & \tiny{  0.035}\\
  3 & \tiny{  0.008} &   0.580 & \tiny{  0.008} &  48 & \tiny{  0.009} &   0.633 & \tiny{  0.009} &  93 & \tiny{  0.013} &   0.831 & \tiny{  0.013} & 138 & \tiny{  0.036} &   1.493 & \tiny{  0.037}\\
  4 & \tiny{  0.008} &   0.580 & \tiny{  0.008} &  49 & \tiny{  0.009} &   0.636 & \tiny{  0.009} &  94 & \tiny{  0.013} &   0.838 & \tiny{  0.013} & 139 & \tiny{  0.037} &   1.522 & \tiny{  0.038}\\
  5 & \tiny{  0.008} &   0.581 & \tiny{  0.008} &  50 & \tiny{  0.009} &   0.638 & \tiny{  0.009} &  95 & \tiny{  0.013} &   0.846 & \tiny{  0.013} & 140 & \tiny{  0.039} &   1.552 & \tiny{  0.040}\\
  6 & \tiny{  0.008} &   0.581 & \tiny{  0.008} &  51 & \tiny{  0.009} &   0.641 & \tiny{  0.009} &  96 & \tiny{  0.013} &   0.854 & \tiny{  0.014} & 141 & \tiny{  0.040} &   1.583 & \tiny{  0.041}\\
  7 & \tiny{  0.008} &   0.581 & \tiny{  0.008} &  52 & \tiny{  0.009} &   0.643 & \tiny{  0.009} &  97 & \tiny{  0.014} &   0.861 & \tiny{  0.014} & 142 & \tiny{  0.042} &   1.615 & \tiny{  0.043}\\
  8 & \tiny{  0.008} &   0.582 & \tiny{  0.008} &  53 & \tiny{  0.009} &   0.646 & \tiny{  0.009} &  98 & \tiny{  0.014} &   0.870 & \tiny{  0.014} & 143 & \tiny{  0.044} &   1.648 & \tiny{  0.045}\\
  9 & \tiny{  0.008} &   0.582 & \tiny{  0.008} &  54 & \tiny{  0.009} &   0.649 & \tiny{  0.009} &  99 & \tiny{  0.014} &   0.878 & \tiny{  0.014} & 144 & \tiny{  0.046} &   1.683 & \tiny{  0.047}\\
 10 & \tiny{  0.008} &   0.582 & \tiny{  0.008} &  55 & \tiny{  0.009} &   0.652 & \tiny{  0.009} & 100 & \tiny{  0.014} &   0.886 & \tiny{  0.014} & 145 & \tiny{  0.048} &   1.719 & \tiny{  0.049}\\
 11 & \tiny{  0.008} &   0.583 & \tiny{  0.008} &  56 & \tiny{  0.009} &   0.655 & \tiny{  0.009} & 101 & \tiny{  0.014} &   0.895 & \tiny{  0.015} & 146 & \tiny{  0.050} &   1.756 & \tiny{  0.052}\\
 12 & \tiny{  0.008} &   0.583 & \tiny{  0.008} &  57 & \tiny{  0.009} &   0.658 & \tiny{  0.009} & 102 & \tiny{  0.015} &   0.904 & \tiny{  0.015} & 147 & \tiny{  0.053} &   1.795 & \tiny{  0.054}\\
 13 & \tiny{  0.008} &   0.584 & \tiny{  0.008} &  58 & \tiny{  0.009} &   0.661 & \tiny{  0.009} & 103 & \tiny{  0.015} &   0.913 & \tiny{  0.015} & 148 & \tiny{  0.055} &   1.836 & \tiny{  0.057}\\
 14 & \tiny{  0.008} &   0.584 & \tiny{  0.008} &  59 & \tiny{  0.009} &   0.664 & \tiny{  0.010} & 104 & \tiny{  0.015} &   0.923 & \tiny{  0.015} & 149 & \tiny{  0.058} &   1.878 & \tiny{  0.060}\\
 15 & \tiny{  0.008} &   0.585 & \tiny{  0.008} &  60 & \tiny{  0.009} &   0.667 & \tiny{  0.010} & 105 & \tiny{  0.015} &   0.933 & \tiny{  0.016} & 150 & \tiny{  0.061} &   1.922 & \tiny{  0.063}\\
 16 & \tiny{  0.008} &   0.586 & \tiny{  0.008} &  61 & \tiny{  0.009} &   0.670 & \tiny{  0.010} & 106 & \tiny{  0.016} &   0.943 & \tiny{  0.016} & 151 & \tiny{  0.065} &   1.968 & \tiny{  0.067}\\
 17 & \tiny{  0.008} &   0.586 & \tiny{  0.008} &  62 & \tiny{  0.010} &   0.674 & \tiny{  0.010} & 107 & \tiny{  0.016} &   0.953 & \tiny{  0.016} & 152 & \tiny{  0.068} &   2.016 & \tiny{  0.071}\\
 18 & \tiny{  0.008} &   0.587 & \tiny{  0.008} &  63 & \tiny{  0.010} &   0.677 & \tiny{  0.010} & 108 & \tiny{  0.016} &   0.964 & \tiny{  0.017} & 153 & \tiny{  0.072} &   2.065 & \tiny{  0.075}\\
 19 & \tiny{  0.008} &   0.588 & \tiny{  0.008} &  64 & \tiny{  0.010} &   0.681 & \tiny{  0.010} & 109 & \tiny{  0.016} &   0.975 & \tiny{  0.017} & 154 & \tiny{  0.076} &   2.117 & \tiny{  0.080}\\
 20 & \tiny{  0.008} &   0.589 & \tiny{  0.008} &  65 & \tiny{  0.010} &   0.684 & \tiny{  0.010} & 110 & \tiny{  0.017} &   0.986 & \tiny{  0.017} & 155 & \tiny{  0.081} &   2.170 & \tiny{  0.085}\\
 21 & \tiny{  0.008} &   0.590 & \tiny{  0.008} &  66 & \tiny{  0.010} &   0.688 & \tiny{  0.010} & 111 & \tiny{  0.017} &   0.997 & \tiny{  0.017} & 156 & \tiny{  0.086} &   2.226 & \tiny{  0.090}\\
 22 & \tiny{  0.008} &   0.591 & \tiny{  0.008} &  67 & \tiny{  0.010} &   0.692 & \tiny{  0.010} & 112 & \tiny{  0.017} &   1.009 & \tiny{  0.018} & 157 & \tiny{  0.091} &   2.283 & \tiny{  0.096}\\
 23 & \tiny{  0.008} &   0.592 & \tiny{  0.008} &  68 & \tiny{  0.010} &   0.696 & \tiny{  0.010} & 113 & \tiny{  0.018} &   1.021 & \tiny{  0.018} & 158 & \tiny{  0.097} &   2.343 & \tiny{  0.102}\\
 24 & \tiny{  0.008} &   0.593 & \tiny{  0.008} &  69 & \tiny{  0.010} &   0.700 & \tiny{  0.010} & 114 & \tiny{  0.018} &   1.034 & \tiny{  0.019} & 159 & \tiny{  0.103} &   2.405 & \tiny{  0.109}\\
 25 & \tiny{  0.008} &   0.594 & \tiny{  0.008} &  70 & \tiny{  0.010} &   0.704 & \tiny{  0.010} & 115 & \tiny{  0.019} &   1.047 & \tiny{  0.019} & 160 & \tiny{  0.110} &   2.469 & \tiny{  0.117}\\
 26 & \tiny{  0.008} &   0.595 & \tiny{  0.008} &  71 & \tiny{  0.010} &   0.708 & \tiny{  0.010} & 116 & \tiny{  0.019} &   1.060 & \tiny{  0.019} & 161 & \tiny{  0.117} &   2.535 & \tiny{  0.125}\\
 27 & \tiny{  0.008} &   0.596 & \tiny{  0.008} &  72 & \tiny{  0.010} &   0.712 & \tiny{  0.011} & 117 & \tiny{  0.019} &   1.074 & \tiny{  0.020} & 162 & \tiny{  0.125} &   2.603 & \tiny{  0.134}\\
 28 & \tiny{  0.008} &   0.597 & \tiny{  0.008} &  73 & \tiny{  0.010} &   0.717 & \tiny{  0.011} & 118 & \tiny{  0.020} &   1.088 & \tiny{  0.020} & 163 & \tiny{  0.133} &   2.673 & \tiny{  0.144}\\
 29 & \tiny{  0.008} &   0.599 & \tiny{  0.008} &  74 & \tiny{  0.010} &   0.721 & \tiny{  0.011} & 119 & \tiny{  0.020} &   1.103 & \tiny{  0.021} & 164 & \tiny{  0.142} &   2.744 & \tiny{  0.155}\\
 30 & \tiny{  0.008} &   0.600 & \tiny{  0.008} &  75 & \tiny{  0.011} &   0.726 & \tiny{  0.011} & 120 & \tiny{  0.021} &   1.118 & \tiny{  0.021} & 165 & \tiny{  0.151} &   2.816 & \tiny{  0.166}\\
 31 & \tiny{  0.008} &   0.601 & \tiny{  0.008} &  76 & \tiny{  0.011} &   0.730 & \tiny{  0.011} & 121 & \tiny{  0.021} &   1.133 & \tiny{  0.022} & 166 & \tiny{  0.161} &   2.890 & \tiny{  0.178}\\
 32 & \tiny{  0.008} &   0.603 & \tiny{  0.008} &  77 & \tiny{  0.011} &   0.735 & \tiny{  0.011} & 122 & \tiny{  0.022} &   1.149 & \tiny{  0.022} & 167 & \tiny{  0.172} &   2.964 & \tiny{  0.191}\\
 33 & \tiny{  0.008} &   0.604 & \tiny{  0.008} &  78 & \tiny{  0.011} &   0.740 & \tiny{  0.011} & 123 & \tiny{  0.022} &   1.166 & \tiny{  0.023} & 168 & \tiny{  0.183} &   3.037 & \tiny{  0.204}\\
 34 & \tiny{  0.008} &   0.606 & \tiny{  0.008} &  79 & \tiny{  0.011} &   0.745 & \tiny{  0.011} & 124 & \tiny{  0.023} &   1.183 & \tiny{  0.023} & 169 & \tiny{  0.194} &   3.111 & \tiny{  0.218}\\
 35 & \tiny{  0.008} &   0.608 & \tiny{  0.009} &  80 & \tiny{  0.011} &   0.751 & \tiny{  0.011} & 125 & \tiny{  0.024} &   1.200 & \tiny{  0.024} & 170 & \tiny{  0.205} &   3.182 & \tiny{  0.232}\\
 36 & \tiny{  0.008} &   0.609 & \tiny{  0.009} &  81 & \tiny{  0.011} &   0.756 & \tiny{  0.011} & 126 & \tiny{  0.024} &   1.219 & \tiny{  0.025} & 171 & \tiny{  0.217} &   3.252 & \tiny{  0.247}\\
 37 & \tiny{  0.008} &   0.611 & \tiny{  0.009} &  82 & \tiny{  0.011} &   0.761 & \tiny{  0.012} & 127 & \tiny{  0.025} &   1.237 & \tiny{  0.025} & 172 & \tiny{  0.228} &   3.318 & \tiny{  0.262}\\
 38 & \tiny{  0.008} &   0.613 & \tiny{  0.009} &  83 & \tiny{  0.011} &   0.767 & \tiny{  0.012} & 128 & \tiny{  0.026} &   1.257 & \tiny{  0.026} & 173 & \tiny{  0.239} &   3.380 & \tiny{  0.276}\\
 39 & \tiny{  0.008} &   0.614 & \tiny{  0.009} &  84 & \tiny{  0.012} &   0.773 & \tiny{  0.012} & 129 & \tiny{  0.026} &   1.277 & \tiny{  0.027} & 174 & \tiny{  0.249} &   3.436 & \tiny{  0.290}\\
 40 & \tiny{  0.008} &   0.616 & \tiny{  0.009} &  85 & \tiny{  0.012} &   0.778 & \tiny{  0.012} & 130 & \tiny{  0.027} &   1.298 & \tiny{  0.028} & 175 & \tiny{  0.258} &   3.486 & \tiny{  0.302}\\
 41 & \tiny{  0.008} &   0.618 & \tiny{  0.009} &  86 & \tiny{  0.012} &   0.784 & \tiny{  0.012} & 131 & \tiny{  0.028} &   1.319 & \tiny{  0.029} & 176 & \tiny{  0.267} &   3.529 & \tiny{  0.313}\\
 42 & \tiny{  0.009} &   0.620 & \tiny{  0.009} &  87 & \tiny{  0.012} &   0.791 & \tiny{  0.012} & 132 & \tiny{  0.029} &   1.342 & \tiny{  0.030} & 177 & \tiny{  0.273} &   3.564 & \tiny{  0.322}\\
 43 & \tiny{  0.009} &   0.622 & \tiny{  0.009} &  88 & \tiny{  0.012} &   0.797 & \tiny{  0.012} & 133 & \tiny{  0.030} &   1.365 & \tiny{  0.031} & 178 & \tiny{  0.278} &   3.589 & \tiny{  0.329}\\
 44 & \tiny{  0.009} &   0.624 & \tiny{  0.009} &  89 & \tiny{  0.012} &   0.803 & \tiny{  0.012} & 134 & \tiny{  0.031} &   1.389 & \tiny{  0.032} & 179 & \tiny{  0.281} &   3.604 & \tiny{  0.333}\\
 45 & \tiny{  0.009} &   0.627 & \tiny{  0.009} &  90 & \tiny{  0.012} &   0.810 & \tiny{  0.013} & 135 & \tiny{  0.032} &   1.413 & \tiny{  0.033} & 180 & \tiny{  0.282} &   3.609 & \tiny{  0.334}\\
\end{tabular}}
\caption{Deferential latitude correction factor for Venus. $\mu$ is in degrees. Note that $\bar{h}(360^\circ-\mu) = \bar{h}(\mu)$, and $\delta h_{\pm}(360^\circ-\mu) = \delta h_{\pm}(\mu)$. }\label{tlat2v}
\end{table}

\newpage
\begin{table}\centering
\small{ \begin{tabular}{crc}
$F (^\circ)$ & $\beta_0(^\circ)$ &
$F (^\circ)$ \\\hline
&&\\[-1.75ex]
000/180 &  0.000 & (180)/(360)\\
002/178 &  0.093 & (182)/(358)\\
004/176 &  0.186 & (184)/(356)\\
006/174 &  0.279 & (186)/(354)\\
008/172 &  0.372 & (188)/(352)\\
010/170 &  0.464 & (190)/(350)\\
012/168 &  0.556 & (192)/(348)\\
014/166 &  0.646 & (194)/(346)\\
016/164 &  0.736 & (196)/(344)\\
018/162 &  0.826 & (198)/(342)\\
020/160 &  0.914 & (200)/(340)\\
022/158 &  1.001 & (202)/(338)\\
024/156 &  1.087 & (204)/(336)\\
026/154 &  1.171 & (206)/(334)\\
028/152 &  1.254 & (208)/(332)\\
030/150 &  1.336 & (210)/(330)\\
032/148 &  1.416 & (212)/(328)\\
034/146 &  1.494 & (214)/(326)\\
036/144 &  1.570 & (216)/(324)\\
038/142 &  1.645 & (218)/(322)\\
040/140 &  1.717 & (220)/(320)\\
042/138 &  1.788 & (222)/(318)\\
044/136 &  1.856 & (224)/(316)\\
046/134 &  1.922 & (226)/(314)\\
048/132 &  1.986 & (228)/(312)\\
050/130 &  2.047 & (230)/(310)\\
052/128 &  2.105 & (232)/(308)\\
054/126 &  2.162 & (234)/(306)\\
056/124 &  2.215 & (236)/(304)\\
058/122 &  2.266 & (238)/(302)\\
060/120 &  2.314 & (240)/(300)\\
062/118 &  2.359 & (242)/(298)\\
064/116 &  2.401 & (244)/(296)\\
066/114 &  2.441 & (246)/(294)\\
068/112 &  2.477 & (248)/(292)\\
070/110 &  2.511 & (250)/(290)\\
072/108 &  2.541 & (252)/(288)\\
074/106 &  2.568 & (254)/(286)\\
076/104 &  2.592 & (256)/(284)\\
078/102 &  2.613 & (258)/(282)\\
080/100 &  2.631 & (260)/(280)\\
082/098 &  2.646 & (262)/(278)\\
084/096 &  2.657 & (264)/(276)\\
086/094 &  2.665 & (266)/(274)\\
088/092 &  2.670 & (268)/(272)\\
090/090 &  2.672 & (270)/(270)\\
\end{tabular}}
\caption{Epicyclic  ecliptic latitude of Mercury.  The latitude is minus the value shown
in the table if the argument is
in parenthesies. }\label{tlat1mc}
\end{table}

\newpage
\begin{table}\centering
\small{ \begin{tabular}{rrrr|rrrr|rrrr|crrr}
$\mu$ & $\delta h_-$  & $\bar{h}~~~~$ & $\delta h_+$ &
$\mu$ & $\delta h_-$  & $\bar{h}~~~~$ & $\delta h_+$ &
$\mu$ & $\delta h_-$  & $\bar{h}~~~~$ & $\delta h_+$ &
$\mu$ & $\delta h_-$  & $\bar{h}~~~~$ & $\delta h_+$ \\\hline
&&&&&&&&&&&&&&&\\[-1.75ex]
  0 & \tiny{  0.099} &   0.719 & \tiny{  0.137} &  45 & \tiny{  0.110} &   0.765 & \tiny{  0.152} &  90 & \tiny{  0.152} &   0.930 & \tiny{  0.217} & 135 & \tiny{  0.274} &   1.283 & \tiny{  0.451}\\
  1 & \tiny{  0.099} &   0.719 & \tiny{  0.137} &  46 & \tiny{  0.110} &   0.768 & \tiny{  0.153} &  91 & \tiny{  0.153} &   0.935 & \tiny{  0.220} & 136 & \tiny{  0.278} &   1.293 & \tiny{  0.461}\\
  2 & \tiny{  0.099} &   0.719 & \tiny{  0.137} &  47 & \tiny{  0.111} &   0.770 & \tiny{  0.154} &  92 & \tiny{  0.155} &   0.941 & \tiny{  0.222} & 137 & \tiny{  0.282} &   1.303 & \tiny{  0.471}\\
  3 & \tiny{  0.099} &   0.719 & \tiny{  0.137} &  48 & \tiny{  0.111} &   0.772 & \tiny{  0.154} &  93 & \tiny{  0.157} &   0.946 & \tiny{  0.225} & 138 & \tiny{  0.286} &   1.313 & \tiny{  0.481}\\
  4 & \tiny{  0.099} &   0.719 & \tiny{  0.137} &  49 & \tiny{  0.112} &   0.774 & \tiny{  0.155} &  94 & \tiny{  0.158} &   0.952 & \tiny{  0.228} & 139 & \tiny{  0.290} &   1.324 & \tiny{  0.491}\\
  5 & \tiny{  0.099} &   0.720 & \tiny{  0.137} &  50 & \tiny{  0.112} &   0.777 & \tiny{  0.156} &  95 & \tiny{  0.160} &   0.958 & \tiny{  0.231} & 140 & \tiny{  0.295} &   1.334 & \tiny{  0.501}\\
  6 & \tiny{  0.099} &   0.720 & \tiny{  0.137} &  51 & \tiny{  0.113} &   0.779 & \tiny{  0.157} &  96 & \tiny{  0.162} &   0.964 & \tiny{  0.234} & 141 & \tiny{  0.299} &   1.344 & \tiny{  0.512}\\
  7 & \tiny{  0.099} &   0.720 & \tiny{  0.137} &  52 & \tiny{  0.113} &   0.782 & \tiny{  0.158} &  97 & \tiny{  0.164} &   0.970 & \tiny{  0.237} & 142 & \tiny{  0.304} &   1.355 & \tiny{  0.523}\\
  8 & \tiny{  0.099} &   0.720 & \tiny{  0.137} &  53 & \tiny{  0.114} &   0.784 & \tiny{  0.159} &  98 & \tiny{  0.165} &   0.976 & \tiny{  0.240} & 143 & \tiny{  0.308} &   1.365 & \tiny{  0.534}\\
  9 & \tiny{  0.100} &   0.721 & \tiny{  0.137} &  54 & \tiny{  0.115} &   0.787 & \tiny{  0.160} &  99 & \tiny{  0.167} &   0.983 & \tiny{  0.243} & 144 & \tiny{  0.313} &   1.375 & \tiny{  0.546}\\
 10 & \tiny{  0.100} &   0.721 & \tiny{  0.138} &  55 & \tiny{  0.115} &   0.790 & \tiny{  0.161} & 100 & \tiny{  0.169} &   0.989 & \tiny{  0.246} & 145 & \tiny{  0.317} &   1.386 & \tiny{  0.558}\\
 11 & \tiny{  0.100} &   0.722 & \tiny{  0.138} &  56 & \tiny{  0.116} &   0.793 & \tiny{  0.162} & 101 & \tiny{  0.171} &   0.996 & \tiny{  0.250} & 146 & \tiny{  0.322} &   1.396 & \tiny{  0.570}\\
 12 & \tiny{  0.100} &   0.722 & \tiny{  0.138} &  57 & \tiny{  0.117} &   0.795 & \tiny{  0.163} & 102 & \tiny{  0.173} &   1.002 & \tiny{  0.253} & 147 & \tiny{  0.326} &   1.406 & \tiny{  0.582}\\
 13 & \tiny{  0.100} &   0.723 & \tiny{  0.138} &  58 & \tiny{  0.117} &   0.798 & \tiny{  0.164} & 103 & \tiny{  0.175} &   1.009 & \tiny{  0.257} & 148 & \tiny{  0.331} &   1.417 & \tiny{  0.594}\\
 14 & \tiny{  0.100} &   0.723 & \tiny{  0.138} &  59 & \tiny{  0.118} &   0.801 & \tiny{  0.165} & 104 & \tiny{  0.178} &   1.016 & \tiny{  0.261} & 149 & \tiny{  0.335} &   1.427 & \tiny{  0.607}\\
 15 & \tiny{  0.100} &   0.724 & \tiny{  0.138} &  60 & \tiny{  0.119} &   0.804 & \tiny{  0.166} & 105 & \tiny{  0.180} &   1.023 & \tiny{  0.264} & 150 & \tiny{  0.340} &   1.437 & \tiny{  0.620}\\
 16 & \tiny{  0.100} &   0.725 & \tiny{  0.139} &  61 & \tiny{  0.120} &   0.807 & \tiny{  0.167} & 106 & \tiny{  0.182} &   1.030 & \tiny{  0.268} & 151 & \tiny{  0.345} &   1.447 & \tiny{  0.633}\\
 17 & \tiny{  0.101} &   0.725 & \tiny{  0.139} &  62 & \tiny{  0.120} &   0.811 & \tiny{  0.168} & 107 & \tiny{  0.184} &   1.037 & \tiny{  0.272} & 152 & \tiny{  0.349} &   1.457 & \tiny{  0.646}\\
 18 & \tiny{  0.101} &   0.726 & \tiny{  0.139} &  63 & \tiny{  0.121} &   0.814 & \tiny{  0.169} & 108 & \tiny{  0.187} &   1.044 & \tiny{  0.276} & 153 & \tiny{  0.354} &   1.467 & \tiny{  0.659}\\
 19 & \tiny{  0.101} &   0.727 & \tiny{  0.139} &  64 & \tiny{  0.122} &   0.817 & \tiny{  0.170} & 109 & \tiny{  0.189} &   1.052 & \tiny{  0.281} & 154 & \tiny{  0.358} &   1.476 & \tiny{  0.672}\\
 20 & \tiny{  0.101} &   0.728 & \tiny{  0.140} &  65 & \tiny{  0.123} &   0.820 & \tiny{  0.172} & 110 & \tiny{  0.192} &   1.059 & \tiny{  0.285} & 155 & \tiny{  0.363} &   1.486 & \tiny{  0.686}\\
 21 & \tiny{  0.101} &   0.729 & \tiny{  0.140} &  66 & \tiny{  0.124} &   0.824 & \tiny{  0.173} & 111 & \tiny{  0.194} &   1.067 & \tiny{  0.290} & 156 & \tiny{  0.367} &   1.495 & \tiny{  0.699}\\
 22 & \tiny{  0.101} &   0.730 & \tiny{  0.140} &  67 & \tiny{  0.124} &   0.827 & \tiny{  0.174} & 112 & \tiny{  0.197} &   1.074 & \tiny{  0.294} & 157 & \tiny{  0.371} &   1.504 & \tiny{  0.713}\\
 23 & \tiny{  0.102} &   0.731 & \tiny{  0.141} &  68 & \tiny{  0.125} &   0.831 & \tiny{  0.176} & 113 & \tiny{  0.199} &   1.082 & \tiny{  0.299} & 158 & \tiny{  0.376} &   1.513 & \tiny{  0.726}\\
 24 & \tiny{  0.102} &   0.732 & \tiny{  0.141} &  69 & \tiny{  0.126} &   0.835 & \tiny{  0.177} & 114 & \tiny{  0.202} &   1.090 & \tiny{  0.304} & 159 & \tiny{  0.380} &   1.522 & \tiny{  0.739}\\
 25 & \tiny{  0.102} &   0.733 & \tiny{  0.141} &  70 & \tiny{  0.127} &   0.838 & \tiny{  0.179} & 115 & \tiny{  0.205} &   1.098 & \tiny{  0.309} & 160 & \tiny{  0.384} &   1.530 & \tiny{  0.753}\\
 26 & \tiny{  0.102} &   0.734 & \tiny{  0.142} &  71 & \tiny{  0.128} &   0.842 & \tiny{  0.180} & 116 & \tiny{  0.208} &   1.107 & \tiny{  0.315} & 161 & \tiny{  0.388} &   1.538 & \tiny{  0.766}\\
 27 & \tiny{  0.103} &   0.735 & \tiny{  0.142} &  72 & \tiny{  0.129} &   0.846 & \tiny{  0.182} & 117 & \tiny{  0.210} &   1.115 & \tiny{  0.320} & 162 & \tiny{  0.392} &   1.546 & \tiny{  0.779}\\
 28 & \tiny{  0.103} &   0.737 & \tiny{  0.142} &  73 & \tiny{  0.130} &   0.850 & \tiny{  0.183} & 118 & \tiny{  0.213} &   1.123 & \tiny{  0.326} & 163 & \tiny{  0.396} &   1.554 & \tiny{  0.791}\\
 29 & \tiny{  0.103} &   0.738 & \tiny{  0.143} &  74 & \tiny{  0.131} &   0.854 & \tiny{  0.185} & 119 & \tiny{  0.216} &   1.132 & \tiny{  0.331} & 164 & \tiny{  0.399} &   1.561 & \tiny{  0.804}\\
 30 & \tiny{  0.104} &   0.739 & \tiny{  0.143} &  75 & \tiny{  0.132} &   0.858 & \tiny{  0.186} & 120 & \tiny{  0.219} &   1.141 & \tiny{  0.337} & 165 & \tiny{  0.403} &   1.568 & \tiny{  0.816}\\
 31 & \tiny{  0.104} &   0.741 & \tiny{  0.144} &  76 & \tiny{  0.133} &   0.862 & \tiny{  0.188} & 121 & \tiny{  0.223} &   1.149 & \tiny{  0.343} & 166 & \tiny{  0.406} &   1.575 & \tiny{  0.827}\\
 32 & \tiny{  0.104} &   0.742 & \tiny{  0.144} &  77 & \tiny{  0.135} &   0.866 & \tiny{  0.190} & 122 & \tiny{  0.226} &   1.158 & \tiny{  0.350} & 167 & \tiny{  0.409} &   1.581 & \tiny{  0.838}\\
 33 & \tiny{  0.105} &   0.743 & \tiny{  0.145} &  78 & \tiny{  0.136} &   0.871 & \tiny{  0.192} & 123 & \tiny{  0.229} &   1.167 & \tiny{  0.356} & 168 & \tiny{  0.412} &   1.587 & \tiny{  0.849}\\
 34 & \tiny{  0.105} &   0.745 & \tiny{  0.145} &  79 & \tiny{  0.137} &   0.875 & \tiny{  0.193} & 124 & \tiny{  0.232} &   1.176 & \tiny{  0.363} & 169 & \tiny{  0.415} &   1.592 & \tiny{  0.858}\\
 35 & \tiny{  0.105} &   0.747 & \tiny{  0.146} &  80 & \tiny{  0.138} &   0.880 & \tiny{  0.195} & 125 & \tiny{  0.236} &   1.185 & \tiny{  0.370} & 170 & \tiny{  0.417} &   1.597 & \tiny{  0.868}\\
 36 & \tiny{  0.106} &   0.748 & \tiny{  0.146} &  81 & \tiny{  0.139} &   0.884 & \tiny{  0.197} & 126 & \tiny{  0.239} &   1.195 & \tiny{  0.377} & 171 & \tiny{  0.420} &   1.602 & \tiny{  0.876}\\
 37 & \tiny{  0.106} &   0.750 & \tiny{  0.147} &  82 & \tiny{  0.141} &   0.889 & \tiny{  0.199} & 127 & \tiny{  0.243} &   1.204 & \tiny{  0.384} & 172 & \tiny{  0.422} &   1.606 & \tiny{  0.884}\\
 38 & \tiny{  0.106} &   0.752 & \tiny{  0.147} &  83 & \tiny{  0.142} &   0.894 & \tiny{  0.201} & 128 & \tiny{  0.246} &   1.214 & \tiny{  0.392} & 173 & \tiny{  0.424} &   1.610 & \tiny{  0.891}\\
 39 & \tiny{  0.107} &   0.754 & \tiny{  0.148} &  84 & \tiny{  0.143} &   0.899 & \tiny{  0.203} & 129 & \tiny{  0.250} &   1.223 & \tiny{  0.400} & 174 & \tiny{  0.425} &   1.613 & \tiny{  0.897}\\
 40 & \tiny{  0.107} &   0.755 & \tiny{  0.149} &  85 & \tiny{  0.145} &   0.904 & \tiny{  0.206} & 130 & \tiny{  0.254} &   1.233 & \tiny{  0.408} & 175 & \tiny{  0.427} &   1.615 & \tiny{  0.903}\\
 41 & \tiny{  0.108} &   0.757 & \tiny{  0.149} &  86 & \tiny{  0.146} &   0.909 & \tiny{  0.208} & 131 & \tiny{  0.258} &   1.243 & \tiny{  0.416} & 176 & \tiny{  0.428} &   1.618 & \tiny{  0.907}\\
 42 & \tiny{  0.108} &   0.759 & \tiny{  0.150} &  87 & \tiny{  0.147} &   0.914 & \tiny{  0.210} & 132 & \tiny{  0.262} &   1.253 & \tiny{  0.424} & 177 & \tiny{  0.429} &   1.619 & \tiny{  0.911}\\
 43 & \tiny{  0.109} &   0.761 & \tiny{  0.151} &  88 & \tiny{  0.149} &   0.919 & \tiny{  0.212} & 133 & \tiny{  0.265} &   1.263 & \tiny{  0.433} & 178 & \tiny{  0.429} &   1.621 & \tiny{  0.913}\\
 44 & \tiny{  0.109} &   0.763 & \tiny{  0.151} &  89 & \tiny{  0.150} &   0.924 & \tiny{  0.215} & 134 & \tiny{  0.269} &   1.273 & \tiny{  0.442} & 179 & \tiny{  0.430} &   1.621 & \tiny{  0.914}\\
 45 & \tiny{  0.110} &   0.765 & \tiny{  0.152} &  90 & \tiny{  0.152} &   0.930 & \tiny{  0.217} & 135 & \tiny{  0.274} &   1.283 & \tiny{  0.451} & 180 & \tiny{  0.430} &   1.622 & \tiny{  0.915}\\
\end{tabular}}
\caption{Deferential latitude correction factor for Mercury. $\mu$ is in degrees. Note that $\bar{h}(360^\circ-\mu) = \bar{h}(\mu)$, and $\delta h_{\pm}(360^\circ-\mu) = \delta h_{\pm}(\mu)$. }\label{tlat2mc}
\end{table}


