\chapter{Dates and times}
% !TEX root = ./Almagest.tex

\section{Julian and Gregorian calendars}
The Julian calendar was instituted by Julius Caesar in 45 BC. However, it was not correctly implemented until
8 AD. The Julian calendar assumes that the length of a tropical year (that is, the time between successive passages of the Sun through the vernal equinox) is $365.25$ days. 
Unfortunately, the true length of the tropical year is $365.2422$ days. This slight error caused the date of the vernal equinox to regress through the calendar over the course
of many centuries. In order to correct this problem, the Gregorian calendar was promulgated by Pope Gregory XIII in 1582. The new calendar was immediately adopted by all of the Catholic counties in Europe, and was eventually
(in some cases, after a delay of hundreds of years)
also adopted by all of the Protestant countries. According to the Gregorian calendar, the length of a tropical year is $365.2425$ days, which is very close to the correct length. 

\section{Julian day number}
Following modern astronomical practice, we shall specify dates  in this treatise 
by means of {\em Julian day numbers}. According to this
scheme, days are numbered consecutively from January 1, 4713 BC (in the Julian calendar), which
is designated day zero. For instance, October 14, 1066 AD (the Julian date of the battle of Hastings) is
day 2\,110\,701. Each Julian day commences at 12:00 universal time (UT). (Universal time is virtually identical to
Greenwich mean time.)

\section{Determination of Julian day numbers}
The Julian day number of a given day can be determined from Tables~\ref{kt1}--\ref{kt3}.  The
date must be expressed in terms of the Gregorian calendar. 

The procedure is as follows:
\begin{enumerate}
\item Enter the table of century years (Table~\ref{kt1}) with the century
year immediately preceding the  date in question, and take out the tabular
value. If the
century year is marked with a $\dag$, note this fact for use in step 2.
\item Enter the table of years of the century (Table~\ref{kt2}) with the
last two digits of the year in question, and take out the tabular value.
If the century year used in step 1 was marked with a $\dag$, diminish the tabular value
by one day, unless the tabular value is zero. If the year in question
is a leap year,  marked with a $\ast$, note this fact for use in step 3.

\item Enter the table of the days of the year (Table~\ref{kt3}) with the
day in question, and take out the tabular value. If the year in question
is a leap year and the table entry falls after February 28, add
one day to the tabular value. The sum of the values obtained
in steps 1, 2, and 3 then gives the Julian day number of the date in question.
\end{enumerate}

~\\
\noindent {\em Example~1}: June 10, 1992 AD:\\~\\
\begin{tabular}{lrrr}
1.~Century year & $^\dag1900$ &&$2\,415\,020$\\
2.~Year of the century & $^\ast92$ & $33\,603 - 1 =$& $33\,602$\\
3.~Day of the year & June 10 & $161+1 =$ & $162$\\\cline{4-4}
Julian day number&&&$2\,448\,784$\\
&&&\\
\end{tabular}\\
Observe that in step 2 the tabular value has been diminished by 1 because 1900
is a common year (marked with a $\dag$ in Table~\ref{kt1}). In step 3, the
tabular value has been increased by 1 because 1992 is a leap year (marked with a $\ast$ in Table~\ref{kt2}), and the
date falls after February 28.

~\\
\noindent {\em Example~2}: January 18, 1824 AD:\\~\\
\begin{tabular}{lrrr}
1.~Century year & $^\dag1800$ &&$2\,378\,496$\\
2.~Year of the century & $^\ast24$ & $8\,766 - 1=$ & 8\,765\\
3.~Day of the year & January 18 & $18 =$ & $18$\\\cline{4-4}
Julian day number&&&$2\,387\,279$\\
&&&\\
\end{tabular}\\
Observe that in step 2 the tabular value has been diminished by 1 because 1800
is a common year (marked with a $\dag$ in Table~\ref{kt1}). In step 3, the
tabular value has not been increased by 1, despite the fact that 1824 is a leap year (marked with an $\ast$ in Table~\ref{kt2}),  because the
date falls before February 28.

We can specify the time of day (in universal time), as well as the date, 
by means of fractional Julian day numbers. For instance,
$t=2\,448\,784.0$ JD corresponds to 12:00 UT  on  June 10, 1992 AD, whereas
$t=2\,448\,784.5$ JD corresponds to 24:00 UT later the same day.

\section{Tables}

\begin{table}[b]\centering
\begin{tabular}{rr}
$^\dag$1700 &  $2\,341\,972$\\
 $^\dag$1800 &  $2\,378\,496$\\
 $^\dag$1900 & $2\,415\,020$\\
 2000 & $2\,451\,544$\\
 $^\dag$2100 & $2\,488\,069$\\
 $^\dag$2200 & $2\,524\,593$\\
  $^\dag$2300 & $2\,561\,117$
\end{tabular}
\caption {Julian Day Number: Century Years. $^\dag$\,Common years. All years are AD. Table reproduced, with permission, from Evans (1998).}\label{kt1}
\end{table}

\vspace*{1cm}

\begin{table}[h]\centering
\begin{tabular}{rr|rr|rr|rr|rr}
$^\S$0 & $0$ & $^\ast$20 & $7\,305$ & $^\ast$40 & $14$\,610 & $^\ast$60 &$21\,915$ & $^\ast$80 & $29\,220$\\
1 & $336$ & 21 & $7\,671$ & 41 & $14\,976$ & 61 &$22\,281$ & 81& $\,29\,586$\\
2 & $731$ & 22 & $8\,036$ & 42 & $15\,341$ & 62 & $22\,646$ & 82 & $29\,951$\\
3 & $1\,096$ & 23 & $8\,401$ & 43 & $15\,706$ & 63 & $22\,011$& 83 &$ 30\,316$\\
&&&&&&&\\
$^\ast$4& $1\,461$& $^\ast$24& $8\,766$& $^\ast$44 & $16\,071$ & $^\ast$64 & $23\,376$& $^\ast$84 & $30\,681$\\
5 & $1\,827$ & 25 & $9\,132$& 45 & $16\,437$& 65& $23\,742$& 85& $31\,047$\\
6& $2\,192$& 26& $9\,497$& 46& $16\,802$& 66& $24\,107$& 86& $31\,412$\\
7& $2\,557$&27&$9\,862$ & 47 & $17\,167$& 67& $24\,472$& 87& $31\,777$\\
&&&&&&&\\
$^\ast$8 & $2\,922$& $^\ast$28 & $10\,227$ & $^\ast 48$ & $17\,532$ & $^\ast 68$ & $24\,837$& $^\ast$88& $32\,142$\\
9 & $3\,288$& 29 & $10\,593$& 49& $17\,898$& 69& $25\,203$& 89& $32\,508$\\
10& $3\,653$& 30 & $10\,958$& 50& $18\,263$& 70& $25\,568$& 90& $32\,873$\\
11& $4\,018$& 31& $11\,323$& 51& $18\,628$& 71& $25\,933$& 91& $33\,238$\\
&&&&&&&\\
$^\ast$12& $4\,383$& $^\ast$32 & $11\,688$ & $^\ast$52 & $18\,993$ &
$^\ast$72 & $26\,298$ & $^\ast$92 & $33\,603$\\
13 & $4\,749$ & 33 & $12\,054$& 53& $19\,359$ & 73& $26\,664$ & 93& $33\,969$\\
14 & $5\,114$ & 34 & $12\,419$ & 54 & $19\,724$ & 74& $27\,029$ & 94&$34\,334$\\
15& $5\,479$ & 35& $12\,784$& 55 & $20\,089$ & 75& $27\,394$ & 95 & $34\,699$\\
&&&&&&&\\
$^\ast$16 & $5\,844$& $^\ast$36& $13\,149$& $^\ast$56 & $20\,454$& $^\ast$76&$27\,759$ & $^\ast$96& $35\,064$\\
17& $6\,210$& 37& $13\,515$& 57& $20\,820$& 77& $28\,125$& 97& $35\,430$\\
18& $6\,575$& 38& $13\,880$& 58& $21\,185$& 78& $28\,490$& 98& $35\,795$\\
19& $6\,940$& 39& $14\,245$& 59& $21\,550$& 79& $28\,855$& 99&$36\,160$\\
\end{tabular}
\caption{Julian Day Number: Years of the Century. $^\ast$\,Leap year. $^\S$\,Leap year unless century is marked $^\dag$. In centuries marked $^\dag$, subtract one day from the tabulated values for the years 1 through 99.
Reproduced, with permission, from Evans (1998).}\label{kt2}
\end{table}

\newpage
\begin{table}[h]\centering
\begin{tabular}{r|rrrrrrrrrrrr}
Day & Jan & Feb & Mar & Apr & May & Jun & Jul & Aug & Sep & Oct& Nov & Dec\\\hline
&&&&&&&&&&&&\\[-1.5ex]
1 & 1 & 32 & 60 & 91 & 121& 152 & 182& 213& 244 & 274& 305& 335\\
2 & 2 & 33 & 61 & 92 & 122& 153& 183& 214& 245& 275& 306& 336\\
3 & 3&34 & 62& 93& 123& 154& 184& 215& 246& 276& 307& 337\\
4& 4& 35& 63& 94& 124& 155& 185& 216& 247& 277& 308& 338\\
5& 5&36& 64& 95& 125& 156& 186& 217& 248& 278& 309& 339\\
&&&&&&&&&&&&\\
6&6& 37& 65& 96& 126& 157& 187& 218& 249& 279& 310& 340\\
7&7 & 38 & 66 & 97& 127& 158& 188& 219& 250& 280& 311& 341\\
8& 8&39& 67& 98& 128& 159& 189& 220&251& 281& 312& 342\\
9&9&40&68&99&129& 160& 190& 221& 252& 282& 313& 343\\
10&10&41&69&100&130& 161& 191& 222& 253& 283&314&344\\
&&&&&&&&&&&&\\
11&11&42&70& 101& 131& 162& 192& 223& 254& 284& 315& 345\\
12& 12&43&71&102& 132& 163& 193& 224& 255& 285& 316& 346\\
13& 13& 44& 72& 103& 133&164& 194& 225& 256& 286& 317& 347\\
14& 14&45& 73& 104& 134& 165& 195& 226& 257& 285& 318& 348\\
15&15&46& 74& 105& 135& 166& 196& 227& 258& 288& 319& 349\\
&&&&&&&&&&&&\\
16& 16& 47& 75& 106& 136& 167& 197& 228& 259& 289& 320& 350\\
17& 17 & 48& 76& 107& 137& 168& 198& 229& 260& 290& 321& 351\\
18& 18& 49& 77& 108&138& 169& 199& 230& 261& 291& 322& 352\\
19& 19& 50& 78& 109& 139&170& 200& 231& 262& 292& 323& 353\\
20 & 20& 51& 79& 110& 140& 171& 201& 232& 263& 293& 324& 354\\
&&&&&&&&&&&&\\
21 & 21 & 52 & 80 & 111& 141& 172& 202& 233& 264& 294& 325& 355\\
22& 22& 53& 81& 112& 142& 173& 203& 234& 265& 295& 326& 356\\
23& 23& 54& 82& 113& 143& 174& 204& 235& 266&296& 327& 357\\
24& 24& 55& 83& 114& 144& 175& 205& 236& 267& 297& 328& 358\\
25&25&56& 84& 115& 145& 176& 206& 237& 268& 298& 329& 359\\
&&&&&&&&&&&&\\
26& 26& 57& 85& 116& 146& 177& 207& 238& 269& 299& 330& 360\\
27& 27& 58& 86& 117& 147& 178& 208& 239& 270& 300& 331& 361\\
28& 28& 59& 87& 118&148& 179& 209& 240& 271& 301& 332& 362\\
29&29&$\ast$&88&119& 149& 180& 210& 241& 272& 302& 333& 363\\
30&30&&89&120&150& 181& 211& 242& 273& 303& 334& 364\\
31&31&&90&&151&&212&243&&304&&365\\
\end{tabular}
\caption{Julian Day Number: Days of the Year. $\ast$\,In leap year,
after February 28, add 1 to the tabulated value. Reproduced, with permission, from Evans (1998).}\label{kt3}
\end{table}
