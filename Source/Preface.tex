\chapter*{Preface}
\addcontentsline{toc}{chapter}{Preface}
% !TEX root = ./Almagest.tex
This book is devoted to a re-examination of Claudius Ptolemy’s Almagest, one of the foundational scientific works of antiquity. Although often contrasted 
unfavorably with Euclid’s Elements, and frequently dismissed as overly complex and misguided in its geocentric approach, the Almagest remains a remarkable 
achievement:  namely, a mathematically sophisticated and observationally grounded attempt to describe the motions of the heavens.
The aim of the present work is to reassess the scientific merits of Ptolemy’s model by reconstructing it in a modern framework. Using contemporary mathematical 
methods, as well as standard astronomical conventions and terminology, we seek to render the structure and logic of the Ptolemaic system more accessible to the 
modern reader. In doing so, we show that many common criticisms of the Almagest are either overstated or based on misunderstandings, and that, when properly 
interpreted, Ptolemy’s model can be viewed as a reasonably accurate  approximation to the later Keplerian description of planetary motion.
At the same time, this book does not attempt to reproduce every aspect of the original treatise. Rather than revisiting the construction of trigonometric tables or 
ancient observational techniques, we focus on the essential geometric and kinematic ideas underlying the model. Certain known deficiencies in Ptolemy’s system 
are corrected, and, where appropriate, his methods are reformulated in terms of equivalent but more transparent schemes.
The result is a streamlined yet faithful reconstruction of the Ptolemaic system—one that preserves its historical character while demonstrating its continued 
mathematical and scientific interest. It is hoped that this approach will allow readers to better appreciate both the ingenuity of Ptolemy’s achievement and its
 place in the development of astronomical science.\\
 ~\\
 \begin{flushright}
 Richard Fitzpatrick\\[0.5ex]
 {\em The University of Texas at Austin}
 \end{flushright}